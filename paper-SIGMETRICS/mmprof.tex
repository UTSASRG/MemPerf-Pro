% \documentclass[manuscript,screen]{acmart}
\documentclass[acmsmall]{acmart}
%Added by Jin
\settopmatter{printacmref=false} % Removes citation information below abstract
\renewcommand\footnotetextcopyrightpermission[1]{} % removes footnote with conference information in first column
\pagestyle{plain} % removes running headers


\AtBeginDocument{%
  \providecommand\BibTeX{{%
    \normalfont B\kern-0.5em{\scshape i\kern-0.25em b}\kern-0.8em\TeX}}}

%% Rights management information.  This information is sent to you
%% when you complete the rights form.  These commands have SAMPLE
%% values in them; it is your responsibility as an author to replace
%% the commands and values with those provided to you when you
%% complete the rights form.
% \setcopyright{acmcopyright}
% \copyrightyear{2018}
% \acmYear{2018}
% \acmDOI{10.1145/1122445.1122456}

%% These commands are for a PROCEEDINGS abstract or paper.
\acmConference[SIGMETRICS '21]{SIGMETRICS '21: ACM Conference}{June 14--18, 2021}{Beijing, China}
% \acmBooktitle{Woodstock '18: ACM Symposium on Neural Gaze Detection,
%   June 03--05, 2018, Woodstock, NY}
% \acmPrice{15.00}
% \acmISBN{978-1-4503-XXXX-X/18/06}


%%
%% Submission ID.
%% Use this when submitting an article to a sponsored event. You'll
%% receive a unique submission ID from the organizers
%% of the event, and this ID should be used as the parameter to this command.
% \acmSubmissionID{123-A56-BU3}
\acmSubmissionID{271}

%%
%% The majority of ACM publications use numbered citations and
%% references.  The command \citestyle{authoryear} switches to the
%% "author year" style.
%%
%% If you are preparing content for an event
%% sponsored by ACM SIGGRAPH, you must use the "author year" style of
%% citations and references.
%% Uncommenting
%% the next command will enable that style.
%%\citestyle{acmauthoryear}

\usepackage[normalem]{ulem}

\usepackage{times}
\usepackage{datetime}
\usepackage{balance}  % to better equalize the last page
\usepackage{graphics} % for EPS, load graphicx instead
\usepackage{url}
\usepackage{hyperref}
\usepackage{txfonts}
\usepackage{pslatex}    
\usepackage{pifont}
%\usepackage{bbding}
\usepackage{multirow}
\usepackage{makecell}
\usepackage[justification=centering]{caption}
\usepackage{xspace}
\usepackage{comment}
\usepackage{listings}
%\usepackage[section]{placeins}
\usepackage{tikz}
\usepackage{calc}
\usepackage{fancyvrb}
\usepackage{xcolor}
\usepackage{flushend}
\usepackage{wrapfig}
\usepackage{graphicx}
\usepackage{float}

\newcommand{\cmark}{\ding{51}}%
\newcommand{\xmark}{\ding{55}}%

\newcommand{\eurosyssubmissionnumber}{\#271, 18 pages}
\newcommand{\todo}[1]{{\color{red}\bfseries [[#1]]}}
\newcommand{\TP}[1]{{\color{red}\bfseries [[#1]]}}

\newcommand{\specialcell}[2][c]{%
  \begin{tabular}[#1]{@{}c@{}}#2\end{tabular}}
\newcommand{\OB}{\texttt{OpenBSD}}
\newcommand{\DieHarder}{\texttt{DieHarder}}
\newcommand{\DL}{\texttt{DLmalloc}}
\newcommand{\JE}{\texttt{jemalloc}}
\newcommand{\MP}{\texttt{mmprof}}
\newcommand{\RN}[1]{\uppercase\expandafter{\romannumeral #1\relax}}

\begin{document}

\title{mmprof: A General Profiler for Different Memory Allocators}

%%
%% The "author" command and its associated commands are used to define
%% the authors and their affiliations.
%% Of note is the shared affiliation of the first two authors, and the
%% "authornote" and "authornotemark" commands
%% used to denote shared contribution to the research.
\author{Anonymous authors}
 \affiliation{
   \institution{Paper under double-blind review}
 }
\renewcommand{\shortauthors}{Anonymous Author, et al.}

\begin{abstract}
The memory allocator is a critical component that can impact the performance of an application by orders of magnitude. However, none of existing profilers helps understand the performance and memory impact caused by the memory allocator. Therefore, programmers may spend unnecessary effort in investigating applications, even if an issue is primarily caused by an inappropriate allocator. This paper designs the first systematic profiler---\texttt{mmprof}---for different memory allocators, which profiles the performance, memory overhead, scalability, and application-friendliness of a memory allocator. \texttt{mmprof} not only helps programmers identify design issues of an allocator without reinventing the wheels every time, but also benefits normal users by presenting performance and memory issues of an allocator. Based on our extensive evaluation, \texttt{mmprof} helps identify multiple known and unknown design issues in widely-used allocators. 
%Toward this goal, this paper defines a range of metrics useful to the evaluation of an allocator. 
Due to its uniqueness and usefulness, \texttt{mmprof} will be an indispensable complement to existing profilers.
\end{abstract}

%%
%% The code below is generated by the tool at http://dl.acm.org/ccs.cfm.
%% Please copy and paste the code instead of the example below.
%%
% \begin{CCSXML}
% <ccs2012>
%  <concept>
%   <concept_id>10010520.10010553.10010562</concept_id>
%   <concept_desc>Computer systems organization~Embedded systems</concept_desc>
%   <concept_significance>500</concept_significance>
%  </concept>
%  <concept>
%   <concept_id>10010520.10010575.10010755</concept_id>
%   <concept_desc>Computer systems organization~Redundancy</concept_desc>
%   <concept_significance>300</concept_significance>
%  </concept>
%  <concept>
%   <concept_id>10010520.10010553.10010554</concept_id>
%   <concept_desc>Computer systems organization~Robotics</concept_desc>
%   <concept_significance>100</concept_significance>
%  </concept>
%  <concept>
%   <concept_id>10003033.10003083.10003095</concept_id>
%   <concept_desc>Networks~Network reliability</concept_desc>
%   <concept_significance>100</concept_significance>
%  </concept>
% </ccs2012>
% \end{CCSXML}

% \ccsdesc[500]{Computer systems organization~Embedded systems}
% \ccsdesc[300]{Computer systems organization~Redundancy}
% \ccsdesc{Computer systems organization~Robotics}
% \ccsdesc[100]{Networks~Network reliability}


% \keywords{datasets, neural networks, gaze detection, text tagging}

\maketitle

\section{Introduction}

Dynamic memory management plays an important role in the performance of applications, especially on multithreaded programs. 
For performance related to memory uses, some work focuses on improving existing memory allocators. Some focuses on a better memory layout among different elements of the same data structure. 
But there is little work that focuses on the improving the performance by changing the behavior of memory allocations and deallocations. 

\HeapPerf{} tries to identify some places inside applications that can introduce the performance problems. These problems can be solved by changing the behavior of memory allocations, without using the new memory allocator. 
These problems are rarely investigated in the past. The most closest work related to this is to simply record the placement with excessive allocations. However, excessive allocations is a total class of the problems that are investigated in this paper, but the existing work fails to present any detailed idea on the following problems: whether all these excessive memory allocations can be reduced? whether they can improve the performance? how to reduce that? These questions requires highly expertise and large amount of manual effort. Instead, \HeapPerf{} presents more useful idea on these questions, and hope to guide programmers, even non-experts, to solve these problems easily. 

We observe three different patterns that can cause performance problems, or called as anti-patterns~\cite{}. 

The second type is shown as Figure~\ref{}. In this example, there are a number of memory allocations that will allocate small amount of bytes for each one. More particularly, these allocations are inside the same loop, and have the same size. Unfortunately, memory allocators will not precisely allocate the specified size of objects. For example, the \texttt{glibc} allocator will allocate 32 bytes as long as the required size is between 4 bytes and 24 bytes. Based on the explanation of Hoard~\cite{Hoard}, this method helps to manage small sizes of different objects, without introducing too many external fragmentation. However, this also introduces a significant problem on cache inefficiency, since only less than 13\% cache is actually utilized, which can introduce around $8\times$ performance slowdown comparing to the code listed in Figure~\ref{}. 


The second type is shown as Figure~\ref{}, there are a number of unnecessary memory allocations and deallocations. By moving the placement of allocations to outside the loop, we can significantly improve the performance by reducing the overhead related to memory allocations and deallocations. 

The third type is related to the uses of heap variables or stack variables. Some excessive heap objects, if they are turned into stack variables, will have large performance benefit. Comparing to heap objects, the overhead of memory allocations and deallocations can be largely reduced if using stack variables. Also, the stack is typically locate inside the cache, which will have lower access latency. Also, stack variables will have exact size, without the addition of metadata and huge alignment. 
\todo{Whether those variables have been touched only very few times, typically should be putted into the heap objects since they may cause the in-efficient cache utilization as well}.  
 


Heap memory related performance bugs can be from the following categories, if only think about applications. 

\begin{itemize}
\item: Too many allocations and deallocations. The total run time spending in memory allocation and liberation may take up to 30\% execution time\cite{1190248}. 

\item: Too many memory uses: this can actually affect the performance when there are too much memory that has been allocated but not used. This can be caused by memory leaks or too late de-allocation. \todo{Most existing tools focus on the memory leak, but whether there are some tools that can uncover delayed-deallocation?}  

\item: 
\end{itemize}


What we can do for heap memory management?

First, we can point out the unnecessary memory allocations and deallocations. For example, we can malloc a large object, and then assign to different small projects. Some of them may be called inside the internal level of loop functions, we can move up to external loop level. 
By reducing unnecessary memory allocations, we expect to improve the performance. 

Second, we can give a statistics on the life-span of objects. Whether we can find out some problems inside? For example, we can use stack variables instead of heap if some objects are too short-lived, or mostly inside a function call. 

Third, we can actually give the statistics on each callsite. Some callsites may have larger number of allocations. 

Can we evaluate the performance related to heap allocations? For example, how much time is spending on memory allocation. 
We can approximate the time of spending on each allocation. Then we can distribute the time to different statements, just similar to gprof. Then maybe it is obvious that we can reduce the overhead by reducing the memory allocations. Then it is possible that a separate paper by using the 


In the end, although not every interested, it is to check the overhead of every memory allocation on each popular memory allocation. Thus, pointing out that the memory allocation actually should pay attention to the level of stacks. Thus, it is possible that we can design a new memory allocator by reducing the level of memory allocation. This is a reverse to HeapLayer. It is great to have an survey paper on this:

A. How is the overhead of memory allocation in large applications? How we can evaluate it? 
B. How is the overhead that comes from memory management? We evaluate this on some popular benchmarks. 
C. Whether the overhead comes from different cache uses? or other things. 
D. It will shed a light whether we need to re-design the memory allocator. 
It will be a perfect paper for ICSE or SC.

%%%%%%%%%%%%%%%%%%%%%%%
% Possible solutions:
% (1) We will check the malloc and free are allocated in sequence. For example, we are always doing the malloc(8) and free(8) in sequence. Given the number of these allocations is large. Then it is much possible that it is a problem. However, it can be a problem for the performance reason. But we can basically maintain a stack that maintains five possible allocations. 
%% Should we just use a two-phase solution? That is, we can use a hash-table to identify different allocation site with their memory uses: how many times for memory allocations? How many times for related free operations? If there are a lot of memory allocations that are not freed, then it is possible a memory leak. We could also identify whether those memory are actually touched or not by using the watchpoint mechanisms? Also, we may try to check whether memory allocation are in the same sequence, for example, alloc-free-alloc-free, and with the same size. If yes, then it is possible that is unnecessary memory operations. 

%%%%%%%%%%%%%%%%%%%%%%%%%
Typically, I think that mtrace utilizes 

Can we check the example of malloc-free situations?
Can we base on a ``anomaly detection'' but.  
I guess that the memory will be freed before the next allocation. If not, then there is high probability of leaking. If memory allocated is on the same site,  

\section{Motivated Example}
\label{sec:motivation}

 Existing profilers cannot help identify some performance issues caused by the memory allocator. Let us use \texttt{cache-thrash} shown in Figure~\ref{fig:motivation} as an example. For this application, TcMalloc runs around $48\times$ slower than the default Linux allocator. We are using perf, gprof, and Coz to analyze the performance issue. The perf result is shown in Figure~\ref{fig:mot1}. 

The perf tool reports that over 99.81\% of time is actually spent inside the \texttt{worker} function, without pinpointing the real issue. Similarly, the gprof tool actually reports that 100\% of time is spent inside this function. Coz is slightly better, reporting the program lines of exercising the objects with passive false sharing issues, i.e. lines 85-87 of \texttt{cache-thrash.cpp}. Coz further predicts that the performance can be improved up to \textbf{10\%} if these lines are improved by 100\%. However, Coz cannot determine whether these lines are actually caused by passive false sharing issues caused by the corresponding allocators.  

%For this application, TcMalloc actually introduces both active and passive false sharing issue, which is the major cause for this large slowdown.  Similarly, Coz also cannot identify the issue that are caused by hardware contention~\cite{DBLP:conf/osdi/ZhouGMW18}. For this example, the first line of the code that can be improved by $XX$ is actually located in line $XX$ of \texttt{} file. That is also nothing related to the allocator.  

\begin{figure}[!ht]
\centering
\includegraphics[width=0.9\columnwidth]{figures/perf-cache-thrash-tcmalloc}
\caption{Profiling result of perf for \texttt{cache-thrash}, with the TcMalloc allocator. \label{fig:mot1}}
\end{figure}

\begin{comment}
\begin{figure}[!ht]
\centering
\includegraphics[width=0.9\columnwidth]{figures/gprof-cache-thrash-tcmalloc}
\caption{Profiling result of gprof for \texttt{cache-thrash}, with the TcMalloc allocator.\label{fig:mot2}}
\end{figure}    
\end{comment}

In contrast, \MP{} reports this problem as passive false sharing issues caused by the allocator, represented by only one parameter inside its application-friendliness metrics. \MP{} also reports a range of metrics for evaluating an allocator, such as memory overhead.    

\section{Background and Overview}
\label{sec:background}

This section presents the background about memory allocators, and some important factors of allocators. Then the basic idea of \MP{} is presented. 

\subsection{Background of Allocator}

\label{sec:allocator}
Memory allocators are typically responsible for managing virtual memory inside the user space by satisfying memory requests from applications. Since the number of small objects is significantly larger than that of big objects, most allocators utilize different mechanisms to manage small and big objects. For big objects, allocators may obtain a block of memory from the OS directly during the allocation, and then return it to the OS upon the deallocation~\citep{Hoard}. For small objects, allocators may utilize freelists or bitmaps to track freed objects upon deallocations. In order to reduce external fragmentation and encourage memory utilization, memory blocks are managed by size classes, and every allocation will be rounded to its next largest size class.  

Based on the management of small objects, allocators can be further classified into multiple categories, such as sequential, BiBOP, and region-based allocators~\citep{DieHarder, Gay:1998:MME:277650.277748}. Region-based allocators are suitable for special situations in which all allocated objects within the same region are deallocated together~\citep{Gay:1998:MME:277650.277748}, which do not belong to the class of general-purpose allocators. Therefore, \MP{} mainly focuses on the other two categories of allocators, where most popular allocators belong to.

For sequential allocators, subsequent memory allocations are satisfied in a continuous memory block. Typically, a pointer is utilized to track the starting position of available space~\citep{Cling}. After an allocation, the pointer is bumped to the end of the current object, which is why this class is also known as ``bump-pointer allocators.'' For such allocators, objects with different sizes can be allocated continuously. Upon deallocation, a freed object is typically placed into the freelist of its size class. The size information of each object is typically physically placed just prior to the object. Such allocators include the default Linux allocator (originating from dlmalloc~\citep{dlmalloc}) and the Windows allocator~\citep{DieHarder}.  

BiBOP-style allocators, which stands for ''Big Bag of Pages''~\citep{hanson1980}, utilize one or multiple continuous pages that are treated as a ``bag'' used to hold objects of the same size class. The metadata of these heap objects, such as their size and availability information, are typically stored in a separate area. Thus, BiBOP-style allocators improve security and reliability by avoiding metadata corruption caused by buffer overflows. Many performance-oriented allocators, such as TcMalloc~\citep{tcmalloc}, \texttt{jemalloc}~\citep{jemalloc}, Hoard~\citep{Hoard}, Scalloc~\citep{Scalloc}, and most of secure allocators, such as OpenBSD~\citep{openbsd} and DieHarder~\citep{DieHarder}, belong to this category. BiBOP-style allocators may utilize freelists or bitmaps to manage the availability of objects. 
%When using a bitmap, only a single bit is sufficient to track the availability of an object, which may introduce less memory overhead for the metadata, but possibly with a higher performance overhead due to the manipulation of the bitmap and the loss of temporal locality.  


\subsection{Important Metrics of Allocators}

\label{sec:factors}

Every memory allocator has its own design choices. However, they share many similar factors, such as performance, memory, scalability, and application-friendliness characteristics. This section will list important metrics for these factors, each of which will also be reported by \MP{}. 

\subsubsection{Performance}
\label{sec:performance}

The performance of using a memory allocator depends on two aspects, the performance overhead of its memory management operations, as well as its application-friendliness toward a specific application (as discussed in Section~\ref{sec: friendliness}). We will focus on the former item here. 

The performance overhead can be evaluated using the average number of instructions and the average runtime for each allocation and deallocation operation. The average data is more intuitive and understandable than the summary value alone. As described above, a memory allocator has different execution paths for different types of allocations and deallocations. Therefore, \MP{} further differentiates the type of memory operation during profiling, such as new or re-used allocations for small objects, deallocations for small objects, and allocations/deallocations for large objects. By doing this, \MP{} is able to identify an issue inside a particular execution path.
%\MP{} relies on a configuration file to obtain the threshold between small and big objects, and . 

In order to reveal a specific design issue, \MP{} further collects the averaged number of instructions, cache misses, and TLB misses for each operation, which are important factors for diagnosing performance issues. For instance, we use the number of cache misses inside DieHarder to diagnose its design issue for its memory deallocations. 
%a large number of instructions possibly indicates an inefficient design of an allocator. 

\subsubsection{Memory Consumption}
\label{sec:memoryconsumption}

Memory consumption is a serious concern across different platforms. Therefore, it is important to assess the memory consumption and wastage of a memory allocator. Sometimes, an allocator may waste more memory than that caused by memory leaks within an application. Based on our understanding, the memory consumption of an allocator may be attributed to multiple sources. First, it may originate from the metadata, such as the memory used to track the size or availability of every heap object. 

Second, it may be caused by internal fragmentation due to the use of size classes.  The difference between the requested size and the size of the corresponding class represents its internal fragmentation, and this space cannot be utilized to satisfy other allocation requests. For instance, Hoard manages objects using power-of-two size classes~\citep{Hoard}, which thus may waste memory if the requested size is not an exact power-of-two. 
%As discussed above, memory allocators utilize multiple size classes to manage heap objects rather than using an exact size.
 
Third, memory allocator wastage may come from ``memory blowup.'' Memory blowup occurs when memory deallocations from one heap cannot be utilized to satisfy subsequent memory requests from another thread~\cite{Hoard}. 
%In order to solve this issue, Li et. al. employ heuristics to adjust the synchronization frequency dynamically~\cite{DBLP:conf/iwmm/LiLD19}. 
Since modern allocators typically utilize per-thread heaps or multiple arenas to reduce the contention overhead, memory blowup is a major source of memory consumption. However, it is challenging to actually quantify memory blowup, as further described in Section~\ref{sec:profilingmemory}.   

Fourth, memory consumption may come from external fragmentation. External fragmentation occurs when the total amount of the available memory is sufficient to satisfy a request but fails to do so due to  non-contiguous memory. External fragmentation is also related to the use of size classes, 
%because allocators rarely or never perform object coalescing and splitting.
since it is typically impossible to change the size class.  
%for a few objects inside the bag, since BiBOP-style allocators typically use a single size for all objects in the entire bag. 
%That is, objects cannot be changed to other size classes, until all objects in the whole bag are freed. 
%This design may cause extensive external fragmentation overhead. 

 Lastly, some secure allocators may voluntarily skip certain objects in order to tolerate buffer overflows~\citep{DieHard, DieHarder, Guarder}. If a buffer overflow lands within these non-used objects, it will cause no harm to the application. However, it is extremely difficult to differentiate between explicit skipping and external fragmentation, with the effects of both ultimately being identical with regard to their impact on this type of memory overhead. 

Overall, for memory consumption, \MP{} individually reports the number and ratio of real memory usage, internal fragmentation, memory blowup, and other overhead. Other overhead includes external fragmentation, metadata overhead, and the sum of skipped objects. It is difficult for \MP{} to possess the precise information about the metadata and the sum of skipped objects. Therefore, \MP{} reports a summary for other memory overhead by subtracting the mentioned values from the total memory usage. For the total memory usage, \MP{} tracks memory-related system calls (e.g., such as \texttt{mmap} or \texttt{sbrk}). 
%Overall, \MP{} provides real memory usage of t
%\MP{} also reports real memory usage by tracking memory allocations and deallocations.  


\subsubsection{Scalability} 
\label{sec:scalability}

The scalability of an allocator can be affected by both hardware and software contention. Hardware contention is mostly related to cache or page contention, which is discussed in Section~\ref{sec: friendliness}. Software contention is the focus here, further including user space contention and kernel contention. 

\paragraph{User Space Contention} The user space contention of an allocator is typically caused by the use of locks inside memory management operations. Based on our observation, different allocators have significantly differing behaviors regarding lock usage. Some allocators, such as TcMalloc~\citep{tcmalloc} or jemalloc~\citep{jemalloc}, minimize the use of locks via per-thread cache. If an allocation can be satisfied from a per-thread cache, there is no need to acquire a lock. However, some allocators, such as Hoard~\citep{Hoard}, acquire at a lock for each allocation request, although using its per-thread heap.p. Some allocators, such as DieHarder and OpenBSD, utilize a central heap (and lock) for each size class, causing too much contention. 

%The average time for each lock acquisition indicates potential lock contention inside. The average time of each critical section helps expose whether the lock contention is due to the heavy workload inside the critical section or not. For instance, if the contention is high, but the average time inside the critical section is low, then this allocator should employ more fine-grained locks to distribute its overhead. In contrast, if the average time inside the critical section is high, then the allocator should possibly move some computation out of the critical section or simplify its management. 

To evaluate user space contention, \MP{} collects per acquisition data, per-operation data, per-lock data, and total information of locks. Per-acquisition data includes the runtime of each lock acquisition and the runtime of each critical section. For each operation, \MP{} reports the number of locks and the average runtime of each acquisition for different operations, such as new small allocations, re-used small allocations, small deallocations, and large allocations and deallocations. Further, \MP{} reports the total number of locks used inside the allocator, then reports the number of acquisitions, the contention rate, and the number of acquisitions for each operation for every suspicious lock that may have a contention issue. 
 
\paragraph{Kernel Space Contention} 
 An allocator may introduce kernel contention by invoking memory-related system calls frequently, such as \texttt{mmap}, \texttt{munmap}, \texttt{madvise}, and \texttt{mprotect}. These system calls may conflict with each other and with the page fault handler. By examining the source code of the Linux kernel, they all acquire a process-based lock (e.g. \texttt{mmap\_sem}) upon the entry of these system calls, causing kernel contention. Based on our evaluation, a version of the Linux allocator slows down an application by more than 20\%, due to extensive invocations of the \texttt{madvise} system call. Therefore, it is important to measure kernel space contention caused by a memory allocator. \MP{} proposes to utilize the average runtime of each system call to evaluate potential kernel contention, without the need to change the kernel source code.

For kernel contention, \MP{} reports the average time and the number of invocations for each memory-related system call. Since different operations (e.g., small versus large allocation) have different execution paths, \MP{} further reports the number for each particular operation. This differentiation helps identify an issue that may only appear in a particular operation pathway. 

\subsubsection{Application Friendliness}
\label{sec: friendliness}

Application friendliness indicates whether memory allocations are suited toward the access patterns of a particular application. Sometimes, application friendliness may have a larger impact on an application's performance than its memory management performance. For instance, TcMalloc typically has less memory management overhead than the default allocator, but runs around $48\times$ slower for \texttt{cache-thrash}. The major reason for this is that TcMalloc introduces both active and passive false-sharing~\cite{tcmallocsharing}. \MP{} reports multiple important metrics that evaluate application-friendliness.


The first parameter measured is the cache utilization rate. The cache utilization rate is the percentage of words that are currently holding actual objects. An allocator with a high cache utilization rate will cause less cache misses, benefiting the overall performance. Multiple causes may affect the cache utilization rate. First, some allocators (e.g., the Linux allocator) that prepend the metadata just prior to each object may reduce the cache utilization rate. Every cache load operation will load the metadata that is not referenced during normal memory access. 
 Second, a coarse-grained size class may also harm the cache utilization rate, due to internal fragmentation. Third, freed objects that are not reutilized in a timely manner may also cause a lower cache utilization rate. 

 %Similarly, if page utilization rate is low, it may cause high TLB misses and prohibitive memory consumption. \MP{} samples memory accesses, and checks the corresponding cache utilization rate and page utilization rate. Overall, \MP{} could report an average cache utilization rate and page utilization rate over all samples. 

The second parameter reported is the page utilization rate. The page utilization rate indicates the percentage of pages that are actively utilized for holding actual data. An allocator with a higher page utilization rate will introduce less page faults and less Translation Lookaside Buffer (TLB) misses. Lower page utilization rates can be caused by reasons similar to those that lower the cache utilization rate.  

The third parameter is the active/passive false sharing. False sharing indicates that multiple threads are concurrently accessing different words within the same cache line. Active false sharing is introduced upon the first allocations of objects, where an allocator cannot allocate continuous objects in the cache line to the same thread. Passive false sharing is introduced upon deallocations, where a freed object will be utilized by another thread, causing false sharing within the same cache line. 

The fourth parameter is the cache contention rate outside of the allocation. Cache misses can be caused by conflicting or falsely-shared misses. Upon cache misses, the data has to be loaded from the main memory, which is significantly slower than accessing the cache directly. 

Overall, \MP{} reports the cache utilization rate, page utilization rate, cache contention rate, and false sharing effect. For the false sharing effect, \MP{} not only reports the number of cache lines that have active and passive false sharing, but also reports the rate of conflicting accesses. The reported result also helps explain the performance slowdown issue. 
%employs the PMU hardware to collect these parameters. It employs the PMU-based sampling to sample memory accesses, and collects the data of cache and page utilization data upon sample events, as described in Section~\ref{sec:profilefriendliness}. \MP{} employs PMU to collect cache misses/page faults outside memory management.

\subsubsection{Summary of Important Metrics}

As described above, \MP{} reports many important metrics of an allocator, as shown in Table~\ref{table:metrics}. 
%These metrics help answer whether an allocator is the culprit of a performance issue, and helps identify a particular design issue within the allocator. 
Table~\ref{table:metrics} also provides collection techniques for specific metrics, which are further discussed in Section~\ref{sec:idea}. 

\begin{table}[h]
  \centering
  \caption{Important metrics and corresponding collection techniques.\label{table:metrics}}
  \footnotesize
 % \setlength{\tabcolsep}{1.0em}
\begin{tabular}{l | l | l}
\hline
Category & Important Metrics & Collection Techniques \\ \hline
\multirow{2}{*}{Performance} & {Alloc/Free runtime} & Timestamp\\ \cline{2-3}
& {Cache misses, page faults, TLB misses, instructions} & Performance Monitoring Units (PMU) \\ \hline
\multirow{4}{*}{Memory} & Internal fragmentation & \\ \cline{2-3}
	& Memory blowup &  \\ \cline{2-3}
& {Other overhead (e.g., external fragmentation)} &  \\ \cline{2-3}
& Real memory usage & \\ \hline
\multirow{2}{*}{Scalability} & \specialcell{User space contention: per-lock data} & Timestamp\\ \cline{2-3}
& {Kernel space contention: per-syscall data} &  Timestamp \\ \hline
\multirow{3}{*}{\specialcell{Application \\ Friendliness}} & Cache/page utilization rate & PMU  \\ \cline{2-3}
& False sharing effect &  PMU\\ \cline{2-3}
& Cache contention rate &  PMU \\ \hline
  \end{tabular}
\end{table}




\section{Overview of mmprof}

\MP{} aims to identify all important factors of memory allocators, as described in Section~\ref{sec:factors}. In the end, \MP{} presents some potential reasons of issues that could help programmers to solve these issues. This Section talks about the basic idea of \MP{}, technical challenges, and related techniques.  
%\MP{} reports some counter information, such as runtime per allocation/deallocation, memory overhead, or lock contention issues. It could further present some internal reasons of causing the issues, by intercepting synchronizations, system calls, and employing hardware performance counters. For instance, the DieHarder allocator is slow in its deallocation, mainly due to the use of a central lock (therefore causing the high contention), the large number of cache misses, and other reasons (as described in Section~\ref{}). Overall,   %Although \MP{} may employ the binary instrumentation to perform a more detailed profiling, such as identifying the instructions within each allocation or deallocation, the binary instrumentation may impose prohibitive performance overhead. With a high overhead, the profiling results may be significantly skewed, such as the waiting time of each lock acquisition inside the allocation. Instead, \MP{} employs the PMU hardware, RDTSC timestamp hardware, and simple counters together to perform the profiling. 

\subsection{Basic Idea}

\MP{} is designed as a drop-in library that can be simply linked to applications (and allocators) with the preloading mechanism, which does not require the change or the re-compilation of applications and allocators.

Sine \MP{} aims to discover the issues of an  allocator, it intercepts the interactions between applications with the memory allocator, such as invocations of memory-related APIs, such as \texttt{malloc}, \texttt{free}, \texttt{calloc}, as shown in Fig.~\ref{fig:basicidea}. Also, it also intercepts the interactions between the allocator with other components of the system, such as the OS (e.g., memory related system calls), and the pthreads library (e.g., synchronizations). By intercepting system calls, it could collect whether an allocator causes significant kernel contention. From synchronizations, it could tell the scalable design of an allocator. As shown in Fig.~\ref{fig:basicidea}, \MP{} also employs hardware Performance Monitoring Units (PMU) to collect hardware related information inside and outside the allocation/deallocation, such as hardware instructions, cache misses, or TLB misses. %All the information are integrated together in order to help programmers identify the particular design issues inside the allocator.  
 
\begin{figure}[!ht]
\centering
\includegraphics[width=3.5in]{figures/overview}
\caption{Basic idea of \texttt{mmprof}.\label{fig:basicidea}}
\end{figure}


\begin{table}[h]
  \centering
  \footnotesize
 % \setlength{\tabcolsep}{1.0em}
\begin{tabular}{l | l | l}
\hline
Category & Collected Data & Collection Methods \\ \hline
\multirow{2}{*}{Performance Overhead} & {Alloc/Free runtime} & Simple Counter, RDTSC Timestamp\\ \cline{2-3}
& {Cache misses, page faults, TLB misses, instructions} & Performance Monitoring Units (PMU) \\ \hline
\multirow{3}{*}{Memory Overhead} & Internal fragmentation & Simple counter\\ \cline{2-3}
	& Memory blowup & Simple counter \\ \cline{2-3}
& {External fragmentation \& others} & Simple counter \\ \hline
\multirow{2}{*}{Scalability} & Lock's acquisition, contention rate, runtime & Simple counter, RDTSC timestamp\\ \cline{2-3}
& {Number and runtime of memory related syscalls} & Simple counter, RDTSC timestamp \\ \hline
\multirow{2}{*}{Application Friendliness} & Cache/page utilization rate & Simple counter, PMU  \\ \cline{2-3}
& \todo{Active/Passive False Sharing} & Simple counter \\ \cline{2-3}
& Cache misses, page faults, TLB misses & PMU\\ \hline
  \end{tabular}
  \centering
  \caption{Profiling data and methods of \MP{}.\label{table:alldata}}
\end{table}

Overall, \MP{} focuses on the allocator itself and its potential impact on applications (called as ``\textit{application friendliness}''). It  collects all data listed in Table~\ref{table:alldata}. The data, when combined together, explains the performance difference of different allocators on the same applications. \MP{} assists allocator developers to identify the potential design issue, without writing a custom profiler. It could also help users to choose an appropriate allocator for a specific application. In the remainder of this section, we describe the idea of profiling different important factors. 


%\MP{} profiles the performance overhead, the memory overhead, and the scalability issues of allocators. Additionally, it also identifies whether an allocator is performance-friendly to a specific application, called as ''\textit{application friendliness}''.  

\subsubsection{Performance Overhead}

For the performance overhead, \MP{} collects the average data of each allocation and deallocation, instead of the summarized values over the whole execution. \MP{} further differentiates the runtime for different types of allocations, such as new/re-used allocation of small and large objects. Each allocation/deallocation data helps identify potential issues inside, if the data is counterintuitive. \MP{} collects the following performance data. 

First, \MP{} collects the runtime of each allocation and deallocation with the RDTSC instruction.  \MP{} obtains the timestamp before and after each operation, and uses the difference of timestamps as the runtime of a specific operation. Its implementation is further described in Section~\ref{sec:performanceimplement}. 

Second, \MP{} also collects hardware events for each allocation and deallocation, such as cache misses, page faults, TLB misses, and instructions, by employing the PMU hardware. These events enable the capability to discover a specific design issue without explicit instrumentation. For instance, a large number of instructions possibly indicates a suceptible design of the allocator. 

\subsubsection{Memory Overhead}

The memory overhead of an allocator comes from multiple aspects, including metadata overhead, internal fragmentation, external fragmentation, memory blowup, or explicit objects skipping. 

The metadata is used to track available objects of each size class and the size information of each object. Based on our understanding, the metadata information is relatively small comparing to other memory overhead, which is omitted by \MP{}. 

Internal fragmentation is introduced when \MP{} uses size classes for allocations to enable memory re-utilization, instead of exact sizes. The difference between the requested size and the size of the corresponding class is internal fragmentation, which cannot be utilized to satisfy other requests. 

External fragmentation occurs, when the total amount of the available memory is sufficient to satisfy a request but fail to do so due to non-contiguous memory. External fragmentation is also related to the use of size classes, because alloctors rarely or never perform objects coalescing and splitting, since it is extremely expensive to perform these operations. Further, it is impossible to change the size class for only few objects inside the bag, since BiBOP-style allocators typically use a single size for all objects in the whole bag. That is, objects cannot be changed to other size classes, until all objects in the whole bag are freed. This design may cause extensive external fragmentation overhead. 

Memory blowup occurs due to the use of multiple heaps for the scalability purpose, where memory deallocations from one heap can not be utilized to satisfy the concurrent memory requests of another heap~\cite{Hoard}. In order to solve this issue, Li et. al. employ heuristics to adjust the synchronization frequency dynamically~\cite{DBLP:conf/iwmm/LiLD19}. Since modern allocators typically utilize per-thread heaps or multiple arenas to reduce the contention overhead,  memory blowup is a big source of memory consumption. It is challenging to profile memory blowup, as further described in Section~\ref{sec:profilingmemory}. 

Secure allocators typically skip some objects in order to improve the security or reliability of the heap~\cite{DieHarder, Guarder}. For instance, if a buffer overflow lands on non-used objects, it will cause no harm to the applications. However, it is extremely difficult to differentiate explicit skipping and external fragmentation. 


%It is relatively easy to compute the alignment overhead, as far as the information of size classes is known, which \MP{} utilizes a pre-run program to obtain (as described in Section~\ref{sec:understandingallocators}). \MP{} tracks each memory allocation, and then computes the alignment overhead for each allocation request. \todo{Of course, upon the deallocation, the corresponding alignment overhead will be extracted from the current overhead. But how we could know the actual alignment overhead for each object? It seems that we should store the information of such objects, or we could recompute due to the last object. }

%\todo{For the memory blowup overhead, \MP{} profiles two types of blowup overhead. One type is simply based on the size of freed objects. If the total size of freed objects is larger than the requested size, but an allocation is satisfied from never-allocated objects, which is consider to be a memory blowup. Another type is based on the total size of freed objects with the same size. } 

%However, it is extremely challenging to compute  the metadata overhead, since it is difficult to identify the location of the metadata, without knowing (or changing) the detailed design of an allocator. \MP{} proposes a novel way to get the metadata overhead, based on the equation~\ref{eq:memoryoverhead}. That is, the metadata overhead can be computed if the total memory overhead ($Total\_{OH}$), alignment overhead ($Align\_{OH}$), or memory blowup ($Blowup$) is known.   

\begin{comment}
\begin{equation}
%\vspace{-0.1in}
\label{eq:memoryoverhead}
Total\_{OH}=Metadata\_{OH}+Align\_{OH}+Blowup
\end{equation} 

That is, we could compute the metadata overhead if we could know the total overhead of the heap, given that $Blowup$ and $Align\_{OH}$ can be computed as described above. The total memory overhead is the difference between the total memory consumption of heap objects and the total requested size of heap objects. For the latter one,  \MP{} could increment the size of every allocation, and then decrement the corresponding size of each deallocation.  That is, the question attributes to the determination of the memory consumption of the heap. We noticed that the \texttt{/proc/PID/smaps} file actually contains the size of physical memory for each virtual memory region (in its \texttt{Referenced} field). That is, we could compute the total physical memory consumption by summing up all physical memory of virtual memory regions that are related to the heap. In order to identify all virtual memory regions belonging to the heap, \MP{} intercepts all memory related system calls, and only includes those ones invoked during allocations and deallocations. 

 


%How we could know the total memory consumption? As described in Section~\ref{}, \MP{} intercepts all memory allocations and deallocations. Also,  all memory-related system calls, such as \texttt{mmap}, \texttt{munmap}, \texttt{madvise}, \texttt{mremap}, and \texttt{sbrk}. occurring inside memory allocations and deallocations will be tracked, where the allocator may utilize the  

For memory overhead, programmers only care about the time with the maximum overhead. To achieve this, \MP{} periodically gets the data about the memory overhead, but only shows the data with the maximum overhead to programmers.  
	
\end{comment}


 
\subsubsection{Scalability Analysis} 
\label{sec:scaleidea}

\MP{} profiles the scalability issues caused by user-space contention and kernel-space contention separately. 

\paragraph{User Space Contention:} User-space contention can be evaluated by explicit uses of locks. \MP{} obtains the number of lock acquisitions, the average time for each lock acquisition, and the average time of spending under the protection of each lock. \MP{} intercepts standard synchronizations in order to collect the data. 
%Therefore, some allocators, such as the default Linux allocators or Hoard, will be changed to use the standard synchronizations.  

The average time for each lock acquisition indicates potential lock contention inside. The average time of each critical section helps expose whether the lock contention is due to the heavy workload inside the critical section or not. For instance, if the contention is high, but the average time inside the critical section is low, then this allocator should employ more fine-grained locks to distribute its overhead. In contrast, if the average time inside the critical section is high, then the allocator should possibly move some computation out of the critical section or simplify its management. 

%\MP{} obtains the average acquisition time for locks without the contention at first. Then it could show whether the lock contention of an allocator is significant high or not.

%At the same time, \MP{} also acquires the time within the critical section. This could help expose whether the lock contention is due to the heavy workload inside the critical section or not. This may require the programmers to take two different actions. 
%In order to reduce unnecessary contention, \MP{} avoids the cache-line based contention. In particular, it utilizes a thread-local pointer that saves the address of thread-local storage. 

\subsubsection{Kernel Contention}
\MP{} also evaluates the potential kernel contention caused by the allocators, since an allocator may interact with the OS substantially. \MP{} does not require to change the kernel directly to achieve this target. Instead, \MP{} monitors the number and the duration time of memory-related system calls inside the user space, such as \texttt{mmap}, \texttt{munmap}, \texttt{madvise}, \texttt{sbrk}, \texttt{madvise}, or \texttt{mremap}. By examining the source code of the Linux kernel, they will acquire a process-based lock (e.g. \texttt{mmap\_sem}) upon the entry of these system calls, causing the kernel-level contention. If an allocator invokes a much larger number of system calls, or if the average time spending on a system call is higher than the execution time of this system call without the contention, which indicating significant contention side, then the allocator should be improved. 

\MP{} intercepts invocations of these system calls, utilizes the RDTSC timestamps to collect the duration time of each system calls, and saves the duration and the number of invocations to the thread-local storage. In the end, it computes the average time and the number of invocations for each system call. Similarly, the average time can be compared with the time without the contention. Therefore, it is easy to know whether there exists kernel contention or not. Therefore, an allocator can be improved by reducing frequent system calls. This method helps to detect an issue of the Linux allocator, which causes significant slowdown on one application. 

  
%\MP{} also tracks the virtual memory regions by analyzing these system calls. 

\begin{comment}
   the information can be utilized to tell whether an allocator has significant  
As we all know, memory allocators may invoke multiple system calls inside allocation and deallocation. 
User space contention:
How many separate locks are explicitly utilized? 
How many lock acquisitions? How much time are spending on lock waiting for each thread, and in total?

How much time spending on kernel-space contention? For instance, we could infer from memory-related system calls, such as mmap, munmap, madvise, brk, or something else? 

That is, we may have to integrate with SyncPerf for doing this. We will borrow their implementation in order to do this. 
\end{comment}

\subsubsection{Application Friendliness} 
\label{sec:friendliness}

Application friendliness helps explain the performance difference of using different allocators. For instance, a well-performed allocator (in its allocation and deallocation) may still greatly slowdown the performance of an application, if it is not cache-friendly. Based on our observation, there are multiple factors that could impact the performance of an application.

The first parameter is the cache utilization rate. Cache utilization rate is the percentage of words that are actually holding actual objects. An allocator with a high cache utilization rate will cause less cache misses, benefiting the overall performance. Multiple reason may affect the cache utilization rate. First, some allocators (e.g., the Linux allocator) that  prepend the metadata just before each object may reduce the cache utilization rate. This design is efficient for its memory management, but reduces the performance for normal memory accesses. Every cache load operation caused by a memory access is forced to load the metadata unnecessarily, which is not used during normal executions. 
 %which could load useful data otherwise.  
 Second, a coarse-grained size class may also reduce cache utilization rate, due to internal fragmentation. Third, freed objects that are not reutilized timely may also cause a low cache utilization rate. 

 %Similarly, if page utilization rate is low, it may cause high TLB misses and prohibitive memory consumption. \MP{} samples memory accesses, and checks the corresponding cache utilization rate and page utilization rate. Overall, \MP{} could report an average cache utilization rate and page utilization rate over all samples. 

The second parameter is page utilization rate. Page utilization rate indicates the percentage of a page that are actively utilized for holding the actual heap data. An allocator with a higher page utilization rate will introduce less page faults and less Translation Lookaside Buffer (TLB) misses. Less page faults and less TLB misses will benefit the performance, since it is  slow to serve a page fault and handle a TLB miss due to the multi-level page table design. Similar to the reasons described above, a low page utilization rate can be caused by the prepending of the metadata, coarse-grained size class, and untimely reutilization of freed objects. 

The third parameter is cache misses of applications. Cache misses can be caused by conflicting or falsely-shared misses. Upon cache misses, the data has to be loaded from the main memory, which is much slower than accessing the cache directly. 

Overall, \MP{} employs the PMU hardware to collect these parameters. It employs the PMU-based sampling to sample memory accesses, and collects the data of cache and page utilization data upon sample events, as described in Section~\ref{sec:profilefriendliness}. \MP{} employs PMU to collect cache misses/page faults outside memory management.



%\todo{cache misses outside allocation and deallocation: }
%\MP{} further checks cache contention rate, which is another important metrics that may significantly affect the performance of applications, although it is not designed as a tool of cache contention detection. For cache contention rate, \MP{} reports the percentage of memory accesses that will cause a cache invalidation.  Similarly, \MP{} also utilizes the sampling mechanism provided by hardware performance counters. Upon each sampled event, \MP{} checks whether the current access causes a cache invalidation or not, and reports the percentage of sampled accesses that could cause the cache invalidation. Basically, \MP{} maintains the cache line ownership for sampled memory accesses, which thread is the last one to write on this cache line. The idea is similar to Cheetah~\cite{Cheetah}. But there are two differences from Cheetah: (1) Cheetah focuses on the identification of false sharing, while \MP{} tries to obtain the basic knowledge about cache contention, including both false sharing and true sharing. During the implementation, Cheetah collects both read/write information of each word, but \MP{} only focuses on the cache line level. (2)  Cheetah only focuses on the cache line with serious issues, which only performs the detection when the number of write accesses on a cache line is over a pre-defined threshold. \MP{} treats every cache line uniformly, and always tracks the cache invalidation information for each cache line. 

%\todo{Maybe we should just use the cache misses of PMUs to evaluate the application friendliness.} We could deduct the number of compulsory cache misses, by checking the region of memory accesses. 

\subsection{Technical Challenges}

As described above, \MP{} employs hardware PMUs, RDTSC, and simple counters together to perform the profiling. However, there exists some technical challenges. 

The first and the most important challenge is the \textbf{overhead challenge}, where a careless design may impose up to 100 $\times$ overhead, based on our past experience of the development. The huge overhead could be unaffordable even for development phases. More importantly, the significant overhead may also skew the evaluation results unnecessarily. 

The second challenge is collect the data precisely, such as memory blowup. Although it sounds intuitive initially to do, it is challenging to deal with the impact of deallocations. 

Other challenges come from the adaption to different allocators. Specific issues include the following ones: (1) How to obtain the specific details of different allocators, such as size class information, type of allocator, metadata size information? (Section~\ref{sec:understandingallocators}) (2) How to design a general but fast lookup mechanism for different allocators? (3) How to profile kernel-contention for allocators?


\subsection{Related Techniques}
\label{sec:pmu}

\paragraph{Hardware Performance Monitor Units (PMUs)} The ubiquitous PMU hardware in modern architectures~\cite{AMDIBS:07, IntelArch:PEBS:Sept09, armpmu} can be employed to sample memory accesses or hardware-related activities~\cite{DBLP:conf/sc/ItzkowitzWAK03, ibs-sc, ibs-pact, Sheng:2011:RLN:1985793.1985848, LASER, Cheetah}. For hardware-related events, it could collect the number of events, such as the number of retired instructions, page faults, TLB load/store misses. Also, it could sample  memory loads and stores, such as IP, timestamps, and memory addresses. Currently, the Linux kernel has supported PMUs starting from 2009 (Linux-2.6.31)~\cite{pmulinuxsupport}, where users could set up performance monitoring via  the \texttt{perf\_event\_open} system call. After collecting events, the user program could fetch these events. 

\paragraph{Time-Stamp Counter} Time-Stamp Counter is a register in all x86 computer that saves the number of cycles since reset. This hardware enables the collection of the time duration accurately with a RDTSC instruction~\cite{coorporation1997using, weaver2013linux}. It has two advantages over system calls like \texttt{gettimeofday}. First, its overhead is much lower than using a system call, typically around 25-35 cycles~\cite{rdtscoverhead}, instead of thousands of cycles. Second, it provides a high-resolution timer, with the granularity of cycles, which is much finer than the microseconds that system calls can provide~\cite{pitfallsrdtsc}. That is helpful to measure the performance of system calls, synchronizations, and memory management operations. 
% Comparing to the method of utilizing system calls, such as \texttt{gettimeofday()}, the RDTSC instruction has two advantages. First, the overhead is much lower than issuing a system call, typically around 25-35 cycles~\cite{rdtscoverhead}. Second, it provides a high-resolution timer, with the granularity of cycles, which is much finer than traditional system calls~\cite{pitfallsrdtsc}. For instance, the system call \texttt{gettimeofday} could only provide the microsecond granularity, which is too coarse for measuring the performance of system calls or synchronization overhead.  






\section{Design and Implementation}
\label{sec:implementation}

This section discusses the implementation and design of \MP{}. It also discusses some common issues, such as adapting to different allocators, and the performance issue of collecting data. \MP{} profiles different aspects of memory allocators, such as performance overhead, memory overhead, scalability issue, and application friendliness. 

\MP{} is implemented as a library that should be preloaded before any other libraries, in order to intercept memory allocations/deallocations, memory-related system calls, and synchronizations. That also indicates that an allocator should utilize standard APIs in order to collect all relevant information. For instance, the Linux allocator utilizes the internal implementation of synchronizations by default.  For the profiling purpose, it should be changed to invoke explicit POSIX-APIs. Fortunately, most allocators do not need any change or the recompilation for the profiling.    

\subsection{Profiling Performance Overhead}

\label{sec:performanceimplement}

\MP{} profiles the performance overhead of every memory management operation. Since an allocator typically has different execution paths for different scenarios (as discussed in Section~\ref{sec:allocator}), such as new or re-used allocations, small or big objects, \MP{} further collects the information for each scenario separately. For allocation, \MP{} collects the data for new allocations of small objects, re-used allocations of small objects, and  allocations of big objects separately. For deallocation, \MP{} collects the deallocation data for small and big objects separately. 


The fine-grained data helps identify a particular design issue inside. For instance, DieHarder was found an serious performance issue for deallocating small objects, but with the normal performance for big objects, as further discussed in Section~\ref{}. Given the fine-grained data, programmers could only focus on the code relating to small objects. \MP{} relies on a configuration file to differentiate small from big objects, as further described in Section~\ref{sec:understandingallocators}, and differentiate re-used allocations from new allocations via a global hash map. All allocated addresses will be inserted into the global hash map.  If the address of an allocation can be found in the global hash map, then it is a re-used allocation. Otherwise, it is a new allocation.  

\MP{} collects the average time of each allocation and deallocation with the RDTSC instruction (Section~\ref{sec:pmu}), due to the performance and accuracy reason. Time stamps are taken before and after each operation, and then the difference between them is  the total number of cycles of an operation. 

\MP{} also employs the PMUs to collect hardware events of each operation, such as cache misses, page faults, TLB misses, and instructions. These hardware data are  significant supplements for identifying an issue, given that \MP{} is a non-intrusive profiler that cannot know the implementation details of an allocator. Let us revisit the issue of DieHarder. DieHarder is found to have a large number of cycles for its deallocation. But it is difficult to know the particular design issue inside, without more information. When given an unusual number of cache misses additionally, the issue related to the cache will be the main focus. Then programmers can easily identify the issue: DieHarder traverses all bags one by one to determine the original bag for an allocation, causing excessive number of cache misses. 

  
\subsection{Profiling Memory Overhead}
\label{sec:profilingmemory}

\MP{} collects real memory usage, real allocated memory, and total memory usage, when the memory usage of an application reaches its maximum value. However, it is very expensive to update these values for every new allocation. Therefore, \MP{} only updates the data when the total memory usage is increased by 1MB. For these data, real memory usage is the amount of memory actually requested by the application, where real allocated memory is the sum of real memory usage and internal fragmentation due to size-class based allocation. For example, if an application requests the memory by \texttt{malloc(454)}, then real memory usage is incremented by $454$, but real allocated memory will be incremented by the size of its corresponding class size. For this example, total memory will be incremented by page size instead (e.g., 4096 bytes), provided that the page was previously unused. 

\MP{} also collects different types of memory overhead, such as internal fragmentation, memory blowup, and others. It also collects the available memory and total memory consumption of each size class. The detailed data help programmers to pinpoint its memory overhead issue. For instance, if memory overhead mainly comes from the internal fragmentation, then the allocator should utilize more fine-grained size classes. If the memory overhead mainly comes from memory blowup, the allocator may require to adjust its synchronization frequency or take a more aggressive method for its synchronization~\citep{DBLP:conf/iwmm/LiLD19}. The memory overhead could be reduced with coalescence or splitting, if the major memory overhead comes from external fragmentation. 

\MP{} typically records different counters upon each allocation and deallocation. For the performance reason, \MP{} typically maintains a per-thread counter in order to reduce the contention issue. The per-thread counters includes the number of allocations and deallocations for each size class, the number of bytes for internal fragmentation, and allocated objects for each size class. 
\MP{} utilizes a global hash table to track the status information for each object, which is the same table for determining new or re-used object. Therefore, \MP{} could adjust the internal fragmentation upon each deallocation. 

\MP{} computes memory blowup for each size class at first, and then summarizes all of them as the total memory blowup. However, it is not intuitive to compute memory blowup. By definition, a new allocation will be treated as memory blowup, if there exist freed objects for this size class in other threads. However, the definition is not clear whether memory blowup should be reduced upon deallocations. \MP{} computes memory blowup for a size class based on the following observation: all freed objects of a size class is the \textit{upper bound} for its memory blowup; this upper bound deducted by recent freed objects will be the real memory blowup. In the implementation, \MP{} utilizes a global counter for each size class to track recent freed objects. This counter is incremented upon every deallocation, but will be decremented upon each allocation \textit{only} when the counter is larger than 0.  

 \MP{} reports other memory overhead as a summarized value, including external fragmentation, metadata overhead, and explicitly skipped objects. It is difficult to differentiate them without the implementation details. %Since \MP{} traps all memory usage requested by the allocator, the remaining memory except internal fragmentation and memory blowup will be reported. 
 

 
 %External fragmentation occurs when an allocator has sufficient memory but in a non-continuous way. Therefore, \MP{} keeps a global counter for the available memory, so that it could determine whether an allocation causes the external fragmentation issue or not.  The counter for the memory blowup will be incremented, if the current per-thread heap has no freed objects but there exit freed objects with the requested size class in other per-thread heaps. Therefore, \MP{} maintains a global counter for each size class. 

% Upon an allocation request, if the current thread does not possess any freed objects of the given size class -- but if the global counter does -- we record this event as an allocation responsible for increasing the allocator's memory blowup.

  
%  \todo{Based on the definitions of memory blowup and external fragmentation, an allocation will be counted as a memory blowup if there exists freed objects with the same size class. However, it is difficult to evaluate the external fragmentation. For instance, if freed objects with smaller class sizes exist, with the total size larger than the requested size, then the current allocation should be counted as external fragmentation. However, if freed objects with larger class sizes exist, it should not count as external fragmentation. But the overhead is caused by the issue of size class or without-splitting. Maybe we should make it clear in Section 3.3. }     
 

\subsection{Profiling Scalability}
\label{sec:profilingscale}

\MP{} also evaluates the scalability of a memory allocator. As described in Section~\ref{sec:scaleidea}, \MP{} evaluates the scalability for both user space and kernel space. 

For user-space scalability, \MP{} focuses on different types of locks, such as mutex locks, spin locks, and try locks. For each lock, \MP{} collects the number of acquisitions, the runtime of each acquisition (with the RDTSC instruction), the runtime of each critical section, the number of contentions, and the number of maximum contending threads. In order to obtain the information of contentions, \MP{} dynamically maintains the contention state of each lock. Upon the acquisition, \MP{} increments the contention threads of the corresponding lock, which will be decremented upon the release of this lock. In theory, the contention data is not accurate, since increments and decrements are not atomic. However,  the data could still show the contenting status of locks inside an allocator.    

For the kernel-space scalability, \MP{} focuses on memory-related system calls inside the allocation and deallocation, including \texttt{mmap}, \texttt{munmap}, \texttt{mremap}, \texttt{sbrk}, \texttt{madvise}, and \texttt{mprotect}. \MP{} profiles the runtime of each invocation with the RDSTC instruction, and the number of invocations for each system call. In fact, the simple data could actually uncover serious scalability issues of a memory allocator. For instance, the Linux allocator of \texttt{glibc-2.21} slows down an application by 40\%, which can be uncovered by the excessive number of \texttt{madvise} system calls and a higher runtime for the corresponding \texttt{mmap} and \texttt{mprotect} system call. In order to identify the data in specific execution path, \MP{} also collect the data for each type of allocations and deallocations, similar to the performance overhead of Section~\ref{sec:performanceimplement}.


\subsection{Application Friendliness}
\label{sec:profilefriendliness}

\MP{} reports whether an allocator is friendly to a specific application. Different from all previous parts that only focusing inside memory management operations, application friendliness focuses on outside memory management operations. More specifically, \MP{} focuses on the following parameters, including (1)~the average cache line utilization; (2)~the average page utilization; (3)~the number of cache misses. As discussed in Section~\ref{sec:friendliness}, a user-friendly allocator should have high cache-line and page utilization rates, but with a low number of cache misses.  

\MP{} employs the PMU-based sampling to track the average cache utilize and page utilization. Basically, \MP{} samples memory accesses periodically, with a default sampling period of 5,000. Upon every sampled event, \MP{} collects the number of used bytes for the current cache line and the current page, and also increments the number of sampled cache lines and pages. In the end, \MP{} computes the utilization rate with a simple division. For cache utilization rate, the dividend is the total number of used bytes, and the divisor is the the total number of bytes for these cache lines. The page utilization rate is computed similarly, but focusing on the page level instead. 

However, the challenge is to quickly locate the  metadata for each cache line and each page, since the metadata should be updated upon every allocation and deallocation, and upon every sampled event. During its implementation, \MP{} has tried multiple mechanisms. First, \MP{} designed a red-black tree to hold memory mappings of the heap, and then stored the address of corresponding metadata on the tree node. This mechanism was found to be in-efficient, since some allocators (e.g., OpenBSD) includes thousands of mappings, which may introduce tens of comparisons unfortunately. Second, \MP{} utilized a hash map to store the memory mappings. But it is difficult to decide the right number of buckets for the hash table, where a small number may cause too many conflicts, with significant performance overhead by traversing the link list.  Finally, \MP{} designs a fast lookup mechanism by taking advantage of the vast address space of 64-bits machines, where the detailed design is discussed as follows. 

% Based on our observation, all allocators invoke either \texttt{sbrk} or \texttt{mmap} system calls to obtain the memory from the underlying OS. The address range returned from \texttt{sbrk} is generally lower than 4G, while the range returned by \texttt{mmap} is typically less than 128TB. 
%Given that modern processors typically support 48 bits address space (256 TB), \MP{} employs the last TB (between 255TB and 256TB) of address space to store the meta data of object, with the design illustrated in Figure~\ref{fig:lookup}. 

\textbf{Three-Level Fast Lookup Mechanism:} \MP{} designs a three-level lookup mechanism as illustrated in Fig.~\ref{fig:lookup}, borrowing the idea of multi-level page table design of OS. Basically, a ``MB Mapping'' will be the first level, where the index can be computed simply by dividing an address with 1 megabytes (MB). Each entry of this MB mapping points to all possible pages inside the current 1-megabyte memory. Since one MB memory will have at most 256 pages, with the size of 4KB for each page, each MB entry points to 256 page entries. Similarly, each page entry has the information about used bytes in this page and has a pointer pointing to 64 possible cache entries inside. Based on this design, it takes two steps to get the used bytes for a page, and takes three steps to get the used bytes for the current cache line. Therefore, it has the $\mathcal{O}(1)$ 
complexity to obtain the metadata.  
          
\begin{figure*}[!h]
\centering
\includegraphics[width=5.5in]{figures/lookup}
\caption{Three-level Lookup Mechanism.\label{fig:lookup}}
\end{figure*}

Note that this design is also efficient in memory consumption. If a range of addresses are not used, then there is no need to allocate physical memory for the corresponding page entries and cache entries. This design is able to adapt to different allocators, where memory mappings of a heap is varied from a few to hundreds of thousands, and these mappings can be scattered along the whole address space of a process. To track valid memory mappings dynamically, \MP{} intercepts memory related system calls inside allocations/deallocations, such as \texttt{sbrk}, \texttt{mmap}, \texttt{munmap}, \texttt{mremap}. 

\subsection{Adapting To Different Allocators}
\label{sec:understandingallocators}

\MP{} is designed as a general profiler for sequential and BiBOP-style allocators, as described in Section~\ref{sec:allocator}. The challenge is to adapt  to different allocators. \MP{} interprets a configuration file to identify the details of every allocation in its initialization phase, such as the allocator's style (BiBOP-style versus sequential), sizes of different classes, the threshold of separating small objects from large objects. \MP{} provides a prerun program to understand these details, or programmers could manually specify the details of an allocator. 

In order to identify the style of allocator, the prerun routine will check whether two subsequent allocations with different sizes (small objects, apparently from different size classes) are allocated from the same page or not. If yes, then the allocator is a sequential-style allocator, which is similar to the default Linux allocator. Otherwise, the allocator belongs to a BiBOP-style allocator. 


The second step is to identify the sizes of different size classes. The prerun routine begins by allocating an object of 8 bytes, and continues to allocate additional objects using a stride increase of 8 bytes each time. The determination of size classes depends on the style of an allocator. For BiBOP-style allocators, an allocation with a different size class will be satisfied from a different bag, locating in a different page. For sequential allocators, such as the Linux allocator, the distance between two contiguously-allocated objects (with distinct sizes) is utilized to determine the size class. As shown in Fig.~\ref{fig:sizeclass}, since the distance of $Obj_1$ and $Obj_2$ is the same as the distance of $Obj_2$ and $Obj_3$, then $obj_1$ and $obj_2$ belong to the same size class. For the same reason, we could determine that $Obj_3$ belongs to a different size class.  

\begin{figure}[!ht]
\centering
\includegraphics[width=5in]{figures/sequentialclasssize}
\caption{Determining the size class of a sequential allocator by the distance between continuous allocations. \\The boxes with 10\% dotted pattern are the metadata, and the boxes with diagonal stripes\\ are actual heap objects. The number above every box is the size of the corresponding object. \label{fig:sizeclass}}
\end{figure}


The threshold for big objects are typically detected by checking whether there is an explicit \texttt{mmap} system calls upon the allocation. Typically, most allocators utilize a direct \texttt{mmap} system call to satisfy the allocation for a big object. However, this threshold requires the manual confirmation. 


\section{Experimental Evaluation}
\label{sec:evaluation}

The experimental evaluation will answer the following questions:
\begin{itemize}
\item How is the effectiveness of \MP{}? (Section~\ref{sec:effectiveness}) 	
\item What is the performance overhead of \MP{}? (Section~\ref{sec:perf})
\item What is the memory overhead of \MP{}? (Section~\ref{sec:memory})
\end{itemize}

Experiments were performed on a two-processor machine, where each processor is Intel(R) Xeon(R) Gold 6230. Each processor has 20 cores in total. This machine has 256GB of main memory, 20MB of L2 cache, and 1280KB L1 cache. The underlying OS is Ubuntu 18.04.3 LTS, installed with the Linux-5.3.0-40. All applications were compiled using GCC-7.5.0, with -O2 and -g flags.

\subsection{Effectiveness}
\label{sec:effectiveness}

In order to evaluate the effectiveness, we evaluate \MP{} with five general purpose allocators, e.g. two versions of the Linux allocator (version 2.21 and 2.28), TCMalloc~\citep{tcmalloc}, jemalloc, and Hoard, and two secure allocators, i.e. DieHarder and OpenBSD. These allocators include both sequential and BiBOP-style allocators. Secure allocators were included, since they have their unique memory management policies. 

For the evaluation, we use the default configurations of these allocators. However, we make some changes in order to the interception of synchronizations. Since Linux allocators are included in \texttt{glibc} libraries, they invokes the internal synchronizations as \texttt{lll\_lock}, which cannot be intercepted by \MP{}. They are compiled separately as a stand-alone library. Since Hoard are using \texttt{std::lock\_guard} for its synchronization, which cannot be intercepted, we replaced them with POSIX spin locks to track its synchronization behavior.

%\todo{Let's use a table to list all dramatic difference between these allocators. This gives us some evidence of allocators}
%\subsection{Issues Identified in Different Allocators}


\begin{table}[h]
  \centering
  \footnotesize
  \setlength{\tabcolsep}{0.2em}
\begin{tabular}{l | l | l | l | l}
\hline
Applications & Allocator & Abnormal Metrics & Possible Root Cause \\ \hline
cache-thrash & TcMalloc & $47.7\times$ slowdown & Contention rate for PFS lines: 50\% & Root Cause \RN{1} \\ \hline
dedup & glibc-2.21 & 20\% slowdown &  \# of Madvise for small allocations: & Root Cause \RN{2} \\ \hline
freqmine & jemalloc &  Memory consumption & memory blowup: 2174230K (37\%) & Root Cause \RN{3} \\ 
 & &  & external fragmentation: 1132045K (19\%) \\ \hline
swaptions  &  DieHarder  & $9\times$ slowdown  & 
	Small reused alloc: 377476 cycles 	& Root Cause \RN{4} \\
& & & Small free: 331745 cycles, and 4.9 cache misses & \\ 
& & & Per-lock acquisition: 353448 cycles & \\\cline{4-5}
& & & External fragmentation: 1878K(37\%) & Root Cause \RN{5} \\
& & & cache utilization 55\%, page utilization 35\% & \\\cline{2-5}
& Hoard & $6.3\times$ slowdown & 
	Small reused alloc: 68933 cycles, 9.4 cache misses 	& Root Cause \RN{6} \\
	
& & & Small free: 53402 cycles, 11.8 cache misses & \\ 
& & & 1.54 locks per-operation & \\
\cline{4-5}
& & & Memory blowup: 4789K( 81\%) & Root Cause \RN{7} \\
& & & cache utilization 62\%, page utilization 51\% & \\\cline{2-5}
& OpenBSD & $8\times$ slowdown & Small re-used alloc: 98962 cycles, 4.5 cache misses & Root Cause \RN{8}\\ 
& & & Small free: 101081 cycles, 7 cache misses & \\ 
& & & 1.1 locks per-operation &  \\ \hline


%\multirow{2}{*}{Performance} & {Alloc/Free runtime} & Timestamp\\ \cline{2-3}

  \end{tabular}
  \centering
  \caption{Abnormal metrics of allocators for different applications.\label{table:abnormal}}
\end{table}

Due to the space limit, we only select multiple examples with abnormal metrics of allocators for the analysis. We aim to cover all allocators, with their abnormal data as shown in Table~\ref{table:abnormal}. Based on these listed metrics, we will show the helpful guidelines provided by \MP{} when analyzing the performance and memory issue. In the end of this section, we also provide some observations based on the evaluation of these allocators.  

 \paragraph{TcMalloc:}
\texttt{TcMalloc} typically performs very well in almost all applications, except for few synthetic applications, such as \texttt{cache-thrash}, \texttt{cache-scratch}, and \texttt{threadtest}. 

\textit{Root Cause \RN{1}}:
For \texttt{cache-thrash}, it runs around $47.7\times$ slower compared to the default Linux allocator. Using \MP{}, we find that the runtime of allocations and deallocations of TcMalloc is actually at a normal range. The only obvious issue is that it has around a 50\% cache contention rate for cache lines with passive false sharing issues, which is the major reason causing the significant slowdown. By checking the source code, we observe that TcMalloc will actually experience both active and passive false sharing issues. For active false sharing, TcMalloc will get one object for a thread from its central heap, so that two continuous objects can be utilized by two different threads. Since TcMalloc always places a freed object to the current thread's per-thread cache, which will also introduce a passive false sharing issue. In comparison, the Linux allocator always returns an object back to its original owner, avoiding passive false sharing.  

\paragraph{glibc-2.21:}
\textit{Root Cause \RN{2}}: The allocator of glibc-2.21 has a bug that invokes excessively large number of \texttt{madvise} systems calls under certain memory use patterns~\cite{madvise}, which is exhibited clearly when running the dedup application. \MP{} reports around 31218 invocations of \texttt{madvise} per second (with a total of 505773 in 16.2 seconds), and the runtime of each \texttt{madvise} is about $23598$ cycles that is $10\times$ of the normal runtime. This clearly indicates that too many \texttt{madvise} system calls introduce contention inside the kernel. Changing the threshold of \texttt{madvise} improves the performance by 20\%.

\paragraph{jemalloc:} jemalloc typically has good performance, but has greater memory consumption.

\textit{Root Cause \RN{3}}: For \texttt{freqmine} application, jemalloc utilizes 6\% more memory than the default Linux allocator, and 36\% more than TcMalloc. Via the report, we can know that jemalloc introduces around 37\% memory blowup and 19\% of external fragmentation of its total memory consumption. In comparison, TcMalloc only has 1\% memory blowup and 13\% external fragmentation.  

\paragraph{DieHarder:} DieHarder performs much slower than other allocators for many applications, and runs $9\times$ slower than the default Linux allocator for \texttt{swaptions}.

\textit{Root Cause \RN{4}}:Based on evaluation results in Table~\ref{table:abnormal}, DieHarder has multiple design issues. From the runtime and lock-related information, we can determine that this allocator introduces an abnormally high amount of cache misses (4.9) for each deallocation. By examining the code, DieHarder must check all miniheaps to identify whether an object belongs to a particular miniheap. This design is not only very slow, but also introduce multiple cache misses by its search. Also, DieHarder utilizes a central lock for all allocations and deallocations, with four locks in total. This design will introduce large slowdowns for parallel applications, which explains why each lock acquisition will take $353,448$ cycles. 

\texttt{Root Cause \RN{5}}: we also notice that DieHarder introduces external fragmentation, around 37\%. As described before, this also includes the size of skipped objects, which is caused by DieHarder's over-provision allocation mechanism. Since DieHarder will also randomly choose some objects, that is maybe the cause of its low cache utilization and page utilization. 


\paragraph{Hoard:} 
 \texttt{Hoard} is running around $6.3\times$ slower than the default allocator. Based on our analysis, it can be caused by multiple reasons.
 
 \texttt{Root Cause \RN{6}:}
 The output of \MP{} shows that it has a large runtime for each allocation and deallocation, and has around 11.8 cache misses. Also, \MP{} reports that it has 1.54 lock acquisitions per call. Clearly, Hoard has a big issue of using locks. By checking the code, we found that Hoard at least acquires a lock for each allocation and deallocation, which is $95713\times$ more than of locks of TcMalloc. TcMalloc utilizes a per-thread cache that there is no need to acquire the lock if an allocation can be satisfied from the per-thread cache. Instead, by using too many locks, Hoard will introduce more cache misses unnecessarily. Another issue is that Hoard are using so many instructions due to its deep-level of templates. For instance, its per-deallocation will has around 1322 instructions, while TcMalloc only has 73.7 instructions and 222 cycles.  
 
 \texttt{Root Cause \RN{7}:} Hoard also has a big issue of memory blowup, with 81\% memory blowup for \texttt{swaptions}. Also, it also much lower cache utilization and page utilization rate than TcMalloc, where TcMalloc's cache and page utilization rate is 80\% and 73\%. That is, all of these factors of Hoard will contribute to the slowdown on this application.
 
\paragraph{OpenBSD:} \texttt{Root Cause \RN{8}:}  OpenBSD has $8\times$ slowdown for \texttt{swaptions}, comparing to the default allocator. Based on its report, we find out that OpenBSD has the similar issue as Hoard, since it acquires more than one lock for each operation. By checking the code, we find out that OpenBSD has the same global lock for all allocations and deallocations, which is the possible reason for its big slowdown. Also, OpenBSD also has a big cache misses for its re-used allocations and deallocations for small objects, which is possibly another reason why it has a big slowdown. For OpenBSD, we also observe that it has significant big number of instructions than other allocators, which as 430 instructions for deallocating a small object, and 295 instructions for a re-used allocation. This is possibly another reason for its big slowdown. 



%\paragraph{jemalloc:}
%During evaluation, the \texttt{reverse\_index} benchmark was found to perform approximately 21\% slower when paired with \texttt{jemalloc} versus the default Linux allocator. Upon inspection, we find that, with \texttt{jemalloc}, the program exhibited over $2x$ the number of CPU cycles associated with the deallocation execution path, as well as a 34\% increase in critical section duration (i.e., the cycles spent within outermost critical sections).




 
\begin{comment}
\renewcommand{\arraystretch}{1.5}
\begin{table}[!ht]
  \centering
   \caption{Important   Metrics\label{tab:metrics}}
  
    \begin{tabular}{l|l|l|l}
    \hline
\multirow{5}{*} {Performance} & \multirow{3}{*}{Allocation Runtime} & New Allocation  (Small) & 80\\ \cline{3-4}
& & Reallocation  (Small) & 1000 \\ \cline{3-4}
& &  Large Allocation & 1000 \\ \cline{2-4}
& \multirow{2}{*}{Deallocation Runtime} & Small  &  \\ \cline{3-4}
& & Large & 100 \\ \cline{1-4}
    
    \end{tabular}
\end{table}
	
\end{comment}




%For a performant allocator, what's the common things within the average allocator. We could utilize a table to list the average points of each allocator. Potentially, we could utilize these parameters to evaluate a new allocator. 

%For evaluating purpose, we could provide two information, one is the average with all evaluated allocators, another one is to omit one allocator with the lowest scores. 


%It seems that BIBOP style allocators are the trend of allocators, which not only has a better performance overhead on average, but also has better safety by separating the metadata from the actual heap. 



\subsection{Performance Overhead}
\label{sec:perf}

\begin{figure}[!ht]
\centering
\includegraphics[width=5.5in]{figures/perfoverhead}
\caption{Performance overhead of \texttt{mmprof}, normalized to the runtime of default Linux allocator.\label{fig:overhead}}
\end{figure}

We also evaluate the performance overhead of 
\MP{} using PARSEC~\citep{parsec},  Phoenix~\citep{phoenix}, and multiple synthetic applications from Hoard~\cite{Hoard}. The performance overhead can be seen in Figure~\ref{fig:overhead}. 

For this figure, we can see that \MP{} runs $2.6\times$ slower than the default allocator, where only two applications, such as \texttt{histogram} and \texttt{threadtest}, impose over $5\times$ performance overhead. Based on our understanding, the following factors contribute to the performance overhead. 

Upon every allocation and deallocation, \MP{} collects the runtime and acquisition information.

\begin{table}[h]
  \centering
  \footnotesize
  \setlength{\tabcolsep}{0.2em}
\begin{tabular}{l|c|r|r|r|r}
\hline
\multicolumn{1}{c|}{Application} & 
\multicolumn{1}{c|}{Runtime}    & 
\multicolumn{1}{c|}{New Alloc}     & 
\multicolumn{1}{c|}{Reused Alloc}     & 
\multicolumn{1} {c|}{Free}     & 
\multicolumn{1}{c}{Lock Acqs} \\ \hline
  blackscholes 16.7 & 8 & 1 & 7 & 11 \\ \hline   
   bodytrack& 8.5 & 20150 & 460616 & 480765 & 871397 \\ \hline    
   cache-scratch \\ \hline    
   cache-thrash  \\ \hline  
   canneal & 29.4 & 8756242 & 12385221 & 21141462 & 9144714 \\ \hline    
   dedup & 12.7 & 3384984 & 683368 & 1750378 & 4864027 \\ \hline    
   facesim & 159.2 & 953143 & 3955049 & 4094483 & 1678963 \\ \hline    
   ferret & 25.3 & 149680 & 236867 & 415914 & 417370\\ \hline    
   fluidanimate 12.3 & 229912 & 1 & 229913 & 307124 \\ \hline    
   freqmine 20.2 & 1810 & 4 & 1070 & 15926 \\ \hline    
   histogram 0.12 & 2 & 0 & 2 & 3 \\ \hline    
   kmeans 16.4 & 200691 & 533 & 200579 & 303705 \\ \hline    
   larson& & \\ \hline    
   linear_regression 0.3 & 1 & 0 & 1 & 2 \\ \hline    
   matrix_multiply 4.8 & 83 & 0 & 82 & 85 \\ \hline    
   pca 9.2 & 16131 & 29 & 72 & 16466 \\ \hline    
   raytrace 41.1 & 5000115 & 15000100 & 20000172 & 5000240 \\ \hline   
   reverse_index 1.5 & 1632810 & 106173 & 1738982 & 1806110\\ \hline  
   streamcluster 23.5 & 47 & 8798 & 8844 & 17622\\ \hline    
   string_match 0.6 & 8 & 0 & 7 & 10 \\ \hline    
   swaptions 14.5 & 2040 & 47999756 & 48000385 & 48002039\\ \hline    
   threadtest&&\\ \hline    
   vips 6.5 & 8128 & 1420072 & 1428019 & 1526404\\ \hline    
   word_count 1.7 & 1 & 0 & 0 & 2\\ \hline   
   x264 24.2 & 10 & 0 & 9 & 13\\ \hline    \hline 
   
     \hline
  \end{tabular}
  \caption{Characteristics of applications\label{table:characteristics}}
\end{table}
\subsection{Memory Overhead}
\label{sec:memory}

We will evaluate the performance overhead of the profiler itself. 

\begin{table}[!tp]  
\centering    
\caption{Memory consumption of Linux's Default with MMProf \label{tab:memory_consumption}}    \begin{tabular}{|l|r|r|}    
\hline    
Applications &  Default  & With \MP{}\\ \hline   
   blackscholes &628681 &1011985\\ \hline    bodytrack& 33070&213612 \\ \hline    cache-scratch &3450 &152028\\ \hline    cache-thrash&3896&155926\\ \hline    canneal&872241&1934704\\ \hline    dedup&1194100&2393497\\ \hline    facesim&322069&714020\\ \hline    ferret&125513&346869\\ \hline    fluidanimate&231920&558696\\ \hline    freqmine&3513129&5904754\\ \hline    histogram&1376432&1509108\\ \hline    kmeans&22538&203553\\ \hline    larson&345286&435672\\ \hline    linear_regression&5830226&5963057\\ \hline    matrix_multiply&50161&197894\\ \hline    pca&502980&827778\\ \hline    raytrace&1317749&2198701\\ \hline    reverse_index&1147026&1842073\\ \hline    streamcluster&114602&315606\\ \hline    string_match&1636385&1769350\\ \hline    swaptions&7676&160146\\ \hline    threadtest&524588&1142986\\ \hline    vips&94312&283276\\ \hline    word_count&3129&729449\\ \hline    x264&1029130&1178769\\ \hline    \hline 
      Total&{\bf 20930289}&{\bf 32143509}\cr\hline    
   \end{tabular}\end{table}



\begin{comment}

\subsection{Range of Allocator Metrics}
We will provide the metrics to evaluate the allocators, based on the averaged value. 
\todo{What types of metrics should we used? For instance, what type of policy should we used to exclude an allocator, and then get the value of the allocator. 20\%}
We will provide a table that can be utilized to evaluate all future allocators. 


%Jin


\end{comment}


\section{Discussion}
\label{sec:limitation}
\MP{} utilizes the RDTSC instruction to collect the runtime of function invocations, and the hardware Performance Monitoring Units (PMUs) to collect hardware events. Although  both of them are likely to be affected by the scheduling,  due to the fact that different cores have different clocks and different set of PMU events, we have confirmed that the scheduling will not cause big difference on the final results. We compared the results with and without thread binding to reach this conclusion. 

%If a thread is scheduled out by the OS scheduler inside an allocation, the runtime of the current allocation will include the waiting time. Similarly, the PMU events for the current allocation will  include some events not belonging to the current thread. Another issue related with the thread migration is that different hardware cores have their own timestamp registers and hardware events are also specific to hardware cores. Therefore, it is inappropriate to use the difference between two cores.  One possible solution is to bind every thread to a specific core, which avoids the migration. But this will significantly interfere with the execution. However, we believe that \MP{} is able to present reliable results statistically, since the number of such scheduling events is generally much less than the number of allocations and deallocations.   

%\MP{} may not get a precise number for contending threads and recently-freed objects, due to potential race conditions. Let us use a simple example to explain this case. If  a counter can be incremented and decremented by multiple threads concurrently, then it is better to utilize the synchronization for every operation. However, it could impose significant performance overhead due to the overhead of employing additional synchronization. For such case, \MP{}  
\section{Related Work}

\cite{1291361} develops two simply analytical model to evaluate the performance impact on large application, based on an application's interaction with the memory system. The observation is that a regular application has continuous and stride memory accesses, while an irregular application has three memory access types: continuous accesses, accesses within the same L1/L2 cache line, and random accesses. This is actually not that related to our system. 

%Potential projects: is there possible to monitor memory accesses pattern and then report irregular pattern and report problems with call site information. 

\cite{Barroso:1998:MSC:279358.279363}: This paper presents a detailed performance study of three important classes of commercial workloads: online transaction processing (OLTP), decision support systems (DSS), and Web index search.  This study characterizes the memory system behavior of these workloads through a large number of architectural experiments on Alpha multiprocessors augmented with full system simulations to determine the impact of architectural trends. We also identify a set of simplifications that make these workloads more amenable to monitoring and simulation without affecting representative memory system behavior. We observe that systems optimized for OLTP versus DSS and index search workloads may lead to diverging designs, specifically in the size and speed requirements for off-chip caches.

Mtrace~\cite{mtrace} traces the events of \texttt{malloc}, \texttt{realloc}, and \texttt{free}, and then post-proposes them. If we can not find its \texttt{free} operations related to a \texttt{malloc}, then this object is considered to be leaked. However, Mtrace can not report the severity of memory leak problems, although this is an engineering problem. \todo{ How is its performance overhead? Maybe we should compare it with ?}   

Mtrace++~\cite{Lee:2000:DMM:786772.787150} is a source code level instrumentation that traces the memory allocations and deallocations. Mtrace++ identifies originations of allocated memories and life spans of objects. But it cannot point out whether problems may occur inside programs. 

Mprof~\cite{Zorn:1988:MAP:894814} first collects the information about allocations and deallocations and writes the information into a file. In the end, it uses the offline analysis to print out the following information on memory leaks, an allocation bin table, a direct allocation table, and a dynamic
call graph. The allocation bin table provides information about what sizes of objects were allocated, and what program types correspond to the sizes listed, so that it helps programmers recognize which data structures consume the most memory and allows him to optimize the space consumption. 
The direct allocation table shows which functions allocate memory and how much they allocate. Mprof introduces around $2\times$ performance overhead that makes it unsuitable for deployment cases. 


mprof presents some information about data structure, thus users can focus on their attention in the future. But that is different from \HeapPerf{}. \HeapPerf{} points out the problem related to the behavior of memory allocation, in which the performance can be improved by changing the behavior of memory allocations, not internal data structure. 
  


\cite{846583}

\cite{1190248}



WMTrace monitors malloc, calloc, realloc, and free events, together with their timing information~\cite{Perks:2011:WAP:2186355.2186369}. In order to reduce the memory consumption and storage consumption, WMTrace{} utilizes the Z-lib to compress the trace, and writes to storage files at the same time as the monitoring. 
 

\subsection{Memory Leak Detector}

LeakPoint\cite{Clause:2010:LPC:1806799.1806874} shares the similar target as \HeapPerf{} on memory leak detection that memory leaks of larger size
are more important than leaks of smaller area. However, LeakPoint imposes between $100\times$ and $300\times$ overhead, while only less than 5\% overhead for \HeapPerf{}. It do not focus on another source of performance problems, those unnecessary memory allocations and deallocations. 


%\section{Allocators}

\subsection{Hoard}

Hoard has multiple sizes: size is less than 256 (Tiny), between 256 and 8192 (Small), between 8193 and 112120064 (Big: larger than 1G). If it is larger than 112120064, we will call it as Huge. 
%If size is less than 256, then the management is Array<NumBins, HL::SLList> _localHeap.
The code can be seen in ThreadLocalAllocationBuffer (LargestObject is 256). The corresponding code can be seen as: 
\begin{lstlisting}
Hoard::ThreadLocalAllocationBuffer<11, &HL::bins<Hoard::HoardSuperblockHeader<HL::SpinLockType, 65536, Hoard::SmallHeap>, 65536>::getSizeClass, &HL::bins<Hoard::HoardSuperblockHeader<HL::SpinLockType, 65536, Hoard::SmallHeap>, 65536>::getClassSize, 256ul, 2097152ul, Hoard::HoardSuperblock<HL::SpinLockType, 65536, Hoard::SmallHeap>, 65536u, Hoard::HoardHeapType>::malloc

Array<NumBins, HL::SLList> _localHeap;

void * malloc (size_t sz) {
  if (sz <= LargestObject) {
    auto c = getSizeClass (sz);
    auto * ptr = _localHeap(c).get();
    if (ptr) {
      _localHeapBytes -= getClassSize (c);         
      return ptr;
    }

    // Now get the memory from our parent.
    auto * ptr = _parentHeap->malloc (sz);
    return ptr;	
 }
}
\end{lstlisting}

For the type of Tiny objects, the LocalHeapThreshold is 2M (what this means? if not larger than 2M, then we do not clear??), while SuperblockSize is 64K. If the size class is less than 256 bytes, then every size class is power of 2. So 16 bytes will be the class 1, 32 will be 2, 64 will be 3, 128 will be 4, and 256 will be 5. If it can't get the object, then it will also get the memory from its parent. It seems that the tiny object will utilize the BIBOP-style that objects with different size class will be located in a different page. In fact, I believe that the size of each block will 0x10000. That will be 64K. But it seems that it is not using the cache warmup mechanism of TcMalloc. If an object cannot be found in the array heap, then it will get it from its parent heap. 

The parent heap is declaimed as HoardHeapType, where the definition can be seen in hoardtlab.h. As follows: \\

\begin{lstlisting}
class HoardHeapType: 
     public HeapManager<TheLockType, HoardHeap<MaxThreads, NumHeaps>> 
\end{lstlisting}

That is, HoardHeapType is defined as HeapManager<TheLockType, HoardHeap<MaxThreads, NumHeaps>>. Among it, TheLockType will be HL::SpinLockType, as defined in hoard/hoardheap.h. MaxThreads is 2048 threads, and NumHeaps is 128, where all of these are defined in hoardconstants.h. However, there is no definition of malloc function in HeapManager. That indicates that malloc will invoke HoardHeap<MaxThreads, NumHeaps>::malloc() instead. 
HoardHeap is defined in hoardheap.h again, and the definition is as follows: 
\begin{lstlisting}
class HoardHeap :
    public HL::ANSIWrapper<
    IgnoreInvalidFree<
      HL::HybridHeap<Hoard::BigObjectSize,
         ThreadPoolHeap<N, NH, Hoard::PerThreadHoardHeap>,
         Hoard::BigHeap> > > 	
\end{lstlisting}


This gives Hoard the capability to control the execution again. Since HoardHeap does not have the malloc() function, then the control will go to HL::ANSIWrapper. That is, unless we can find an object in ArrayHeap, every allocation will invoke ANSIWrapper::malloc(). Therefore, if the size is larger than 2G, the allocation will fail. If the size is less than 16, it will use 16 as the minimum size class. In this class, the allocation will satisfied by IgnoreInvalidFree class. However, IgnoreInvalidFree does not have the malloc() function either, it will call HL::HybridHeap<Hoard::BigObjectSize,
         ThreadPoolHeap<N, NH, Hoard::PerThreadHoardHeap>,
         Hoard::BigHeap> instead. In HrbridHeap, both small allocation and big allocation will be satisfied from the same heap. 
         Basically, the allocation will be as follows:
         
\begin{lstlisting}
if (sz <= BigSize) {
   ptr = SmallHeap::malloc (sz);
} else {
   ptr = slowPath (sz);
}
\end{lstlisting}

Here, BigSize is defined as 8192. But it is very weird that we can't put any printing in this function. Otherwise, the program will crash. 

\subsubsection{Small Heap}
For small heap, it will invoke ThreadPoolHeap<N, NH, Hoard::PerThreadHoardHeap> to get the object. That is, if the size is less than 8192, it will call ThreadPoolHeap::malloc(). 


\begin{lstlisting}
template <int NumThreads,
      int NumHeaps,
      class PerThreadHeap_>
  class ThreadPoolHeap : public PerThreadHeap_ {	
   inline PerThreadHeap& getHeap (void) {
      auto tid = HL::CPUInfo::getThreadId();
      auto heapno = _tidMap(tid & NumThreadsMask);
      return _heap(heapno);
    }

    inline void * malloc (size_t sz) {
      return getHeap().malloc (sz);
    }
}
\end{lstlisting}

Now we will utilize PerThreadHeap concept here. For getThreadId(), it will invoke the pthread\_self() system call. In total, there are 128 heaps. The heap definition is at Array<MaxHeaps, PerThreadHeap>. Since PerThreadHeap is defined as Hoard::PerThreadHoardHeap, as detailed in the following (in hoardheap.h). That is, we will utilize the tid to get one per-thread heap, and then allocated from the per-thread heap. 

\begin{lstlisting}
  class PerThreadHoardHeap :
    public RedirectFree<LockMallocHeap<SmallHeap>,
      SmallSuperblockType>	
\end{lstlisting}

For LockMallocHeap, the definition will be like this. 
\begin{lstlisting}
class LockMallocHeap : public Heap {
  public:
    MALLOC_FUNCTION INLINE void * malloc (size_t sz) {
      std::lock_guard<Heap> l (*this);
      return Heap::malloc (sz);
    }	
\end{lstlisting}

Basically, this just places a lock before doing the allocation. The real heap is actually SmallHeap as defined in the following:

\begin{lstlisting}
class SmallHeap : 
    public ConformantHeap<
    HoardManager<AlignedSuperblockHeap<TheLockType, SUPERBLOCK_SIZE, MmapSource>,
     TheGlobalHeap,
     SmallSuperblockType,
     EMPTINESS_CLASSES,
     TheLockType,
     hoardThresholdFunctionClass,
     SmallHeap> >	
\end{lstlisting}

Sine ConformantHeap has no definition of malloc, we will invoke HoardManager::malloc() instead. When there is no existing objects in the heap, it will call slowPathMalloc() to do the allocation. slowPathMalloc() actually has a for loop, when the allocation is not successful, it will invoke getAnotherSuperblock() to get a super block. Then the next loop will get one object any way. Then all objects in this super block will be utilized to satisfy requests with the same size. That is, it will invoke MmapSource to get one super block. That is, for every size class of each thread, it will get 64K for each block. 

For deallocation, we will get the superblock for each object at first by invoking SuperHeap::getSuperblock(). It may force every superblock allocation to be aligned to 64 K. Then it could just use the header as the management for the superblock. For each object, it will utilize the Array<NumBins, BinManager> to manage small objects.  In total, there are 11 bins for 64K superblock size. 

But Array do not have free function. In fact, it is defined in BinManager, which is ManageOneSuperblock<OrganizedByEmptiness>. That is, for every operation, it will call OrganizedByEmptiness::SuperblockType. Since OrganizedByEmptiness is defined by EmptyClass<SuperblockType, EmptinessClasses> OrganizedByEmptiness, then the actual free will invoke SmallSuperblockType::free(), and the allocation will invoke SmallSuperblockType::malloc(). But based on the definition, typedef HoardSuperblock<TheLockType, SUPERBLOCK\_SIZE, SmallHeap> SmallSuperblockType. That is, in the end, it will call  HoardSuperblock::free to free the object (\_header.free). \_header is defined as Hoard::HoardSuperblockHeader<LockType, SuperblockSize, HeapType> Header. That is, it will insert the object into the freelist ( a singular linklist). If all objects in the superblock are freed, then we will call clear operation. Basically, it will utilize the original data as the pointers for the linklist.  

For each allocation, it will simply utilize the bump pointer to allocate an object. 

HoardManager will also check the threshold. It will free up a superblock if we've crossed the emptiness threshold. Based on the definition, it will actually invoke hoardThresholdFunctionClass() to check the threshold. Then we will remove a superblock and give it to the 'parent heap', which is a global heap for such size class.  
     
\begin{lstlisting}
bool function (int u, int a, size_t objSize)
    {
      auto r = (u < 0.909 * a)) 
      && ((u < a - (2 * 64K) / objSize));
      return r;
    }
\end{lstlisting}


\subsubsection{Big Heap}

The big heap is defined as following:
\begin{lstlisting}
Hoard::ThreadLocalAllocationBuffer<11, &HL::bins<Hoard::HoardSuperblockHeader<HL::SpinLockType, 65536, Hoard::SmallHeap>, 65536>::getSizeClass, &HL::bins<Hoard::HoardSuperblockHeader<HL::SpinLockType, 65536, Hoard::SmallHeap>, 65536>::getClassSize, 256ul, 2097152ul, Hoard::HoardSuperblock<HL::SpinLockType, 65536, Hoard::SmallHeap>, 65536u, Hoard::HoardHeapType>::malloc

  class HoardHeapType :
    public HeapManager<TheLockType, HoardHeap<MaxThreads, NumHeaps> > {
  };

  class HoardHeap :
    public HL::ANSIWrapper<
    IgnoreInvalidFree<
      HL::HybridHeap<Hoard::BigObjectSize,
         ThreadPoolHeap<N, NH, Hoard::PerThreadHoardHeap>,
         Hoard::BigHeap> > >

typedef HL::ThreadHeap<64, HL::LockedHeap<TheLockType,
              ThresholdSegHeap<25,      // % waste
                   1048576, // at least 1MB in any heap
                   80,      // num size classes
                   GeometricSizeClass<20>::size2class,
                   GeometricSizeClass<20>::class2size,
                   GeometricSizeClass<20>::MaxObjectSize,
                   AdaptHeap<DLList, objectSource>,
                   objectSource> > >
  bigHeapType;
  
  class BigHeap : public bigHeapType {};
  
  typedef HoardSuperblock<TheLockType, SUPERBLOCK_SIZE, BigHeap> BigSuperblockType;
	
\end{lstlisting}


Basically, if the size is larger than 256, then we will invoke the HoardHeapType::malloc() in class ThreadLocalAllocationBuffer. For HoardHeapType::malloc(), it will invoke HoardHeap --> HL::ANSIWrapper --> HL::HybridHeap::malloc. 
In HybridHeap, if the size is larger than 8192, we will invoke bm.malloc. In fact, it will invoke Hoard::BigHeap::malloc(). That is, it will actually bigHeapType::malloc(). 

For ThreadHeap, there are 64 heaps. The source code is as follows:

\begin{lstlisting}
 void * malloc (size_t sz) {
   auto tid = Modulo<NumHeaps>::mod (CPUInfo::getThreadId());
   return getHeap(tid)->malloc (sz);
}	
\end{lstlisting}

For PerThreadHeap, it actually invokes LockedHeap (defined in lockedheap.h). In fact, this will invoke ThresholdSegHeap based on the definition of bigHeapType.  If the size is 9000, then the sizeClass is 29, and maxSize of this sizeClass is 10240. NumBins is 80, which is corresponding to more than 1G's allocation (with the size of 112120064). 
\begin{lstlisting}
if (sizeClass >= NumBins) {
  return BigHeap::malloc (maxSz);
} else {
  void * ptr = _heap[sizeClass].malloc (maxSz);
  if (ptr == NULL) {
    fprintf(stderr, "sz %d maxSz %ld\n", sz, maxSz);
    return BigHeap::malloc (maxSz);
  }	
\end{lstlisting}

 That is, it will invoke \_heap[sizeClass].malloc for normal large allocations (between 8192 and 112120064). 
class objectSource : public AddHeaderHeap<BigSuperblockType,
              SUPERBLOCK\_SIZE,
              MmapSource> {};





For less than 256, we will use per thread cache bins, if the cache has some objects. The code can be seen in in superblocks/tlab.h, around line 90. If not, then we will allocate some objects from per size class. That is, if multiple threads are accessing some objects, they will allocate from the same super-block.  If there is no objects for this size class, then we will \_parentHeap->malloc() to grab one superblock. However, if the objects are not available, we will use per-thread heap. For each size class, we will have a super-block (64K). 

Between 256 and 8192, we will also use power of 2 as size classes. But we will use PerThreadHeap. In fact, the allocation will be called by HoardManager::slowPathMalloc(). In this function, it will call \_otherBins(binIndex).malloc (sz). (Array<NumBins, BinManager>). But in the end, it will call ThreadPoolHeap<N, NH, Hoard::PerThreadHoardHeap>. That is, it will actually call PerThreadHoardHeap::malloc. In the end, it will get a superblock (64K) for each thread.  In fact, there maybe exists a bug for size between 256 and 8192. \todo{Even if there are freed objects in the same size class for the same heap, then the allocation will be satisfied from never-allocated objects (maybe just the last item is not reused). Also, it seems that a thread will get objects in a different thread. For instance, the main thread has two freed objects, then one of them can be utilized by the child thread. But the main thread will not get objects that are just deallocated by its child threads. That is very weird, different threads have the same heapno after the mod operation. But the objects cannot be re-used immediately.  }

If the size is larger than 8192, we will use a class size, and we will only get the object with the size class without using the per-thread heap. For normal large allocations, the code is defined defined in thresholdsegheap.h around line 50.  In stead, we will just use class objectSource : public AddHeaderHeap<BigSuperblockType, AdaptHeap<DLList, objectSource>. Also, it has an issue here, by using the size class incorrectly. For instance, if the allocation is 9000, then the size class will be 10240. However, the actual object will be aligned to 3 pages. Then we will actually invoke mmap() to allocate 3 pages (not aligning to super block).


For big objects, it will handle differently. All big objects will be utilized the same parameters to control, such as cLive, maxLive, and maxFraction. If the size class is larger than 1GB, then the object will be deallocated immediately. Otherwise, the deallocation will be based on the following code. That is, it will check whether the current live is larger than 80\%. mFraction is defined as 25\% and threshSlop is defined as 1MB. That is, if the currently-using memory is less than 1MB, then the freed object will never be returned to the OS. That may indicates that we are not actively using big objects.  If the live memory  (currently using) is less than 4 times of previous maximum live memory, then we will return the memory back to the OS. That is, we are not aggressively using more memory recently, so that we could return all available memory back. There are two issues here. First, why it does not return the memory based on the available memory of the heap? Second, why it will clear all memory in the freelists, why not save some of them for future allocation?

\begin{lstlisting}
 cLive -= sz;
       
 _heap[cl].free (ptr);
 bool thresh = mLive > mFraction * cLive;
 if ((cLive > threshSlop) && thresh && !cleared)
 {
    // Clear the heap.
    for (int i = 0; i < NumBins; i++) {
      _heap[i].clear();
    }
    // We won't clear again until we reach maxlive again.
    cleared = true;
    mLive = cLive;
 }
\end{lstlisting}




\section{Conclusion}
\label{sec:conclusion}

Memory allocator could significantly impact the performance of applications. However, none of the existing profilers helps understand the behavior of an allocator, and determine  a performance or memory overhead issue caused by the allocator. This paper designs the first general profiler to profile the performance, memory overhead, scalability, and application-friendliness of a memory allocator, by the employment of simple counters, timestamp register, and hardware performance monitoring units. It will benefit allocator programmers and normal users. Our extensive evaluation has confirmed that \texttt{mmprof} helps identify multiple design issues in widely-used allocators. Therefore, \texttt{mmprof} will be an indispensable complementary to existing profilers. 

{
\bibliographystyle{plain}
\bibliography{refs, tongping, jin}
}

\end{document}
\endinput
