
\begin{table}[h]
  \centering
  \caption{Abnormal metrics of allocators for different applications.\label{table:abnormal}}
  \footnotesize
  \setlength{\tabcolsep}{0.2em}
\begin{tabular}{l | l | l | l | l}
\hline
Applications & Allocator & Behavior & Abnormal Metrics & Root Cause \\ \hline
cache-thrash & TcMalloc & $47.7\times$ slowdown & Contention rate for PFS lines: 50\% & Root Cause \RN{1} \\ \hline
dedup & glibc-2.21 & 20\% slowdown &  \# of Madvise for small allocations: & Root Cause \RN{2} \\ \hline
freqmine & jemalloc &  Memory consumption & memory blowup: 2174230K (37\%) & Root Cause \RN{3} \\ 
 & &  & external fragmentation: 1132045K (19\%) \\ \hline
swaptions  &  DieHarder  & $9\times$ slowdown  & 
	Small reused alloc: 377476 cycles 	& Root Cause \RN{4} \\
& & & Small free: 331745 cycles, and 4.9 cache misses & \\ 
& & & Per-lock acquisition: 353448 cycles & \\\cline{4-5}
& & & External fragmentation: 1878K(37\%) & Root Cause \RN{5} \\
& & & cache utilization 55\%, page utilization 35\% & \\\cline{2-5}
& Hoard & $6.3\times$ slowdown & 
	Small reused alloc: 68933 cycles, 9.4 cache misses 	& Root Cause \RN{6} \\
	
& & & Small free: 53402 cycles, 11.8 cache misses & \\ 
& & & 1.54 locks per-operation & \\
\cline{4-5}
& & & Memory blowup: 4789K( 81\%) & Root Cause \RN{7} \\
& & & cache utilization 62\%, page utilization 51\% & \\\cline{2-5}
& OpenBSD & $8\times$ slowdown & Small re-used alloc: 98962 cycles, 4.5 cache misses & Root Cause \RN{8}\\ 
& & & Small free: 101081 cycles, 7 cache misses & \\ 
& & & 1.1 locks per-operation &  \\ \hline
  \end{tabular}
\end{table}

Due to the space limit, we only select multiple examples with abnormal metrics of allocators for the analysis. We aim to cover all allocators, with their abnormal data as shown in Table~\ref{table:abnormal}. Based on these listed metrics, we will show the helpful guidelines provided by \MP{} when analyzing the performance and memory issue. In the end of this section, we also provide some observations based on the evaluation of these allocators.  

 \paragraph{TcMalloc:}
TcMalloc typically performs very well in almost all applications, except for few synthetic applications, such as \texttt{cache-thrash}, \texttt{cache-scratch}, and \texttt{threadtest}. 

\textit{Root Cause \RN{1}}:
For \texttt{cache-thrash}, it runs around $47.7\times$ slower compared to the default Linux allocator. Using \MP{}, we find that the runtime of allocations and deallocations of TcMalloc is actually at a normal range. The only obvious issue is that it has around a 50\% cache contention rate for cache lines with passive false sharing issues, which is the major reason causing the significant slowdown. By checking the source code, we observe that TcMalloc will actually experience both active and passive false sharing issues. For active false sharing, TcMalloc will get one object for a thread from its central heap, so that two continuous objects can be utilized by two different threads. Since TcMalloc always places a freed object to the current thread's per-thread cache, which will also introduce a passive false sharing issue. In comparison, the Linux allocator always returns an object back to its original owner, avoiding passive false sharing.  

\paragraph{glibc-2.21:}
\textit{Root Cause \RN{2}}: The allocator of glibc-2.21 has a bug that invokes excessively large number of \texttt{madvise} systems calls under certain memory use patterns~\cite{madvise}, which is exhibited clearly when running the dedup application. \MP{} reports around 31218 invocations of \texttt{madvise} per second (with a total of 505773 in 16.2 seconds), and the runtime of each \texttt{madvise} is about $23598$ cycles that is $10\times$ of the normal runtime. This clearly indicates that too many \texttt{madvise} system calls introduce contention inside the kernel. Changing the threshold of \texttt{madvise} improves the performance by 20\%.

\paragraph{jemalloc:} jemalloc typically has good performance, but has greater memory consumption.

\textit{Root Cause \RN{3}}: For \texttt{freqmine} application, jemalloc utilizes 6\% more memory than the default Linux allocator, and 36\% more than TcMalloc. Via the report, we can know that jemalloc introduces around 37\% memory blowup and 19\% of external fragmentation of its total memory consumption. In comparison, TcMalloc only has 1\% memory blowup and 13\% external fragmentation.  

\paragraph{DieHarder:} DieHarder performs much slower than other allocators for many applications, and runs $9\times$ slower than the default Linux allocator for \texttt{swaptions}.

\textit{Root Cause \RN{4}}:Based on evaluation results in Table~\ref{table:abnormal}, DieHarder has multiple design issues. From the runtime and lock-related information, we can determine that this allocator introduces an abnormally high amount of cache misses (4.9) for each deallocation. By examining the code, DieHarder must check all miniheaps to identify whether an object belongs to a particular miniheap. This design is not only very slow, but also introduce multiple cache misses by its search. Also, DieHarder utilizes a central lock for all allocations and deallocations, with four locks in total. This design will introduce large slowdowns for parallel applications, which explains why each lock acquisition will take $353,448$ cycles. 

\textit{Root Cause \RN{5}}: we also notice that DieHarder introduces external fragmentation, around 37\%. As described before, this also includes the size of skipped objects, which is caused by DieHarder's over-provision allocation mechanism. Since DieHarder will also randomly choose some objects, that is maybe the cause of its low cache utilization and page utilization. 


\paragraph{Hoard:} 
 \texttt{Hoard} is running around $6.3\times$ slower than the default allocator. Based on our analysis, it can be caused by multiple reasons.
 
 \textit{Root Cause \RN{6}:}
 The output of \MP{} shows that it has a large runtime for each allocation and deallocation, and has around 11.8 cache misses. Also, \MP{} reports that it has 1.54 lock acquisitions per call. Clearly, Hoard has a big issue of using locks. By checking the code, we found that Hoard at least acquires a lock for each allocation and deallocation, which is $95713\times$ more than of locks of TcMalloc. TcMalloc utilizes a per-thread cache that there is no need to acquire the lock if an allocation can be satisfied from the per-thread cache. Instead, by using too many locks, Hoard will introduce more cache misses unnecessarily. Another issue is that Hoard are using so many instructions due to its deep-level of templates. For instance, its per-deallocation will has around 1322 instructions, while TcMalloc only has 73.7 instructions and 222 cycles.  
 
 \textit{Root Cause \RN{7}:} Hoard also has a big issue of memory blowup, with 81\% memory blowup for \texttt{swaptions}. Also, it also much lower cache utilization and page utilization rate than TcMalloc, where TcMalloc's cache and page utilization rate is 80\% and 73\%. That is, all of these factors of Hoard will contribute to the slowdown on this application.
 
\paragraph{OpenBSD:} \textit{Root Cause \RN{8}:}  OpenBSD has $8\times$ slowdown for \texttt{swaptions}, comparing to the default allocator. Based on its report, we find out that OpenBSD has the similar issue as Hoard, since it acquires more than one lock for each operation. By checking the code, we find out that OpenBSD has the same global lock for all allocations and deallocations, which is the possible reason for its big slowdown. Also, OpenBSD also has a big cache misses for its re-used allocations and deallocations for small objects, which is possibly another reason why it has a big slowdown. For OpenBSD, we also observe that it has significant big number of instructions than other allocators, which as 430 instructions for deallocating a small object, and 295 instructions for a re-used allocation. This is possibly another reason for its big slowdown. 

\todo{For TcMalloc, maybe we should avoid its active and passive false sharing, then see how much performance improvement on it?}



%\paragraph{jemalloc:}
%During evaluation, the \texttt{reverse\_index} benchmark was found to perform approximately 21\% slower when paired with \texttt{jemalloc} versus the default Linux allocator. Upon inspection, we find that, with \texttt{jemalloc}, the program exhibited over $2x$ the number of CPU cycles associated with the deallocation execution path, as well as a 34\% increase in critical section duration (i.e., the cycles spent within outermost critical sections).




