\subsection{Effectiveness}
\label{sec:effectiveness}

In the following, we evaluate how \MP{} could benefit both normal users and allocator designers. 

\subsubsection{Benefiting Normal Users\\} 
\label{sec:normalusers}

\noindent \textbf{Predicting Performance Impact:} 
\MP{} can predict the performance impact if switching to a new allocator as discussed in Section~\ref{sec:predict}. Here, we are utilizing the average cycles of TcMalloc listed in Table~\ref{tbl:metrics} to predict the performance impact of switching to TcMalloc. All applications listed in Figure~\ref{fig:motivation} are evaluated, except \texttt{cache-scratch},  \texttt{cache-thrash}, and \texttt{freqmine}.  Since \texttt{cache-scratch} and \texttt{cache-thrash} are running much slower with TcMalloc due to its passive/active false sharing issue, their performance results with TcMalloc cannot serve as the baseline correctly. \texttt{freqmine} is an openmp program that \MP{} cannot support well. The prediction results can be seen in Table~\ref{tbl:predictionResult}, where ``reverse'' is the abbreviation of reverse\_index. 

\begin{table}[]
  \centering
  \footnotesize
  \setlength{\tabcolsep}{0.2em}
\begin{tabular}{l|c|c|c|c|c|c|c|c|c|c|c|c}
\hline
 \multirow{2}{*}{APP} &
  \multicolumn{2}{c|}{Default} &
  \multicolumn{2}{c|}{glibc-2.21} &
  \multicolumn{2}{c|}{jemalloc} &
  \multicolumn{2}{c|}{TcMalloc} &
  \multicolumn{2}{c|}{Hoard} &
  \multicolumn{2}{c}{DieHarder} \\ \cline{2-13}
  & R & P & R & P & R & P & R & P & R & P  & R & P    \\ \hline
canneal        & 1.05 & 1.05 & 1.07 & 1.06 & 1.01 & 1.03 & 1.00 & 1.02 & 1.03 & 1.11 & 1.39 & 2.00 \\ \hline
dedup          & 1.06 & 1.01 & 1.35 & 1.01 & 1.06 & 1.00 & 1.00 & 1.00 & 1.02 & 1.00 & 2.91 & 1.89 \\ \hline
%freqmine       & 0.90 & 1.00 & 0.96 & 1.00 & 1.01 & 1.00 & 1.00 & 1.00 & 1.26 & 1.00 & 3.32 & 1.01 \\ \hline
kmeans         & 1.16 & 1.00 & 1.16 & 1.00 & 1.06 & 1.00 & 1.00 & 1.00 & 1.02 & 1.00 & 1.03 & 1.00 \\ \hline
raytrace       & 1.27 & 1.02 & 1.27 & 1.05 & 1.20 & 1.00 & 1.00 & 1.00 & 1.10 & 1.01 & 1.31 & 1.51 \\ \hline
reverse & 1.00 & 1.07 & 0.99 & 1.07 & 1.05 & 1.08 & 1.00 & 1.04 & 1.15 & 1.16 & 2.42 & 1.89 \\ \hline
swaptions      & 0.99 & 0.99 & 0.99 & 1.00 & 0.98 & 0.96 & 1.00 & 0.96 & 2.04 & 1.11 & 5.67 & 3.82 \\ \hline
\end{tabular}
   \caption{ Results of \MP{}'s performance prediction, where the data is normalized to that of TcMalloc. ``R'' and ``P'' columns list real and predicted result.  \label{tbl:predictionResult}}
\end{table}

Overall, \MP{} could successfully predict most serious performance impact (over 20\%) of  applications caused by an allocator (e.g., Hoard and DieHarder), although the exact numbers are different. For instance, \MP{} predicts that switching DieHarder to TcMalloc may boost the performance of \texttt{canneal} by $2\times$, but TcMalloc only improves the performance by 39\% in reality. Multiple reasons can contribute to the prediction difference or inaccuracy. First, \MP{} uses the averaged cycles as the baseline for prediction. But the real runtime can be affected by multiple factors, such as number of page faults, system calls, and lock contention. Some of them (e.g., lock contention) are very difficult to quantify, and thus not considered during the prediction. Second, \MP{} can only predict the impact caused by slow memory management operations, but not on impact caused by false sharing or cache misses. For instance, the runtime of memory management operations in \texttt{kmeans} is only a small portion of the total runtime (less than 3\%), with less than 2000 allocations. That is, different allocators may have different application-friendliness factors that may affect the performance of \texttt{kmeans}, instead of slow memory management operations.  Third, \MP{} assumes no dependency between threads and predicts  based on the maximum parallel phase. However, in \texttt{dedup}, the longest thread was always waiting for other threads, but without allocations and deallocations inside. Although other threads could be significantly improved, \MP{} could not predict much impact on the final performance. To be more accurate, the prediction should consider the dependency between all threads~\cite{wPerf}, but that is too complicated itself to be included here.

\textbf{Reporting Memory Overhead:} 
For each application, \MP{} reports memory overhead caused by the allocator. Here, we list high memory overhead of two industrial allocators. \MP{} reports that \texttt{jemalloc} has high memory overhead in \texttt{freqmine}, \texttt{pca}, \texttt{raytrace} and \texttt{vips}. For these applications, \texttt{jemalloc} consumes 11\% more memory on average, compared to the default Linux allocator. \MP{} also reports that TcMalloc has memory issues in \texttt{swaptions} and \texttt{vips}, where it spends 26\% more memory than the default Linux allocator on average. 
%Take \texttt{raytrace} as an example,  when the allocator is \texttt{jemalloc}, the memory overhead is 10\% more than the default Linux allocator. \MP{} presents \texttt{jemalloc} only effectively utilizes 65\% of its memory consumption, and the other 35\% is the external fragmentation, which indicates \texttt{jemalloc} could have a memory management issue when running \texttt{raytrace}.


%In \texttt{vips}, \MP{} shows the size of objects only occupies 72\% of \texttt{TcMalloc}'s memory consumption, and percentages of internal fragmentation, memory blowup and external fragmentation are 3\%, 12\% and 13\%.

\textbf{Reporting Application Friendliness:} 
\MP{} also reports different application friendliness for each allocator, which helps explain the performance difference between different allocators. 

Let us revisit the example of \texttt{cache-thrash} in Section~\ref{sec:intro}. \MP{} reports that 18\% instructions will cause cache invalidation based on our simulation results for \texttt{TcMalloc}, but there is  cache invalidation for the default allocator. 
%Similar for \texttt{cache-scratch}, \texttt{TcMalloc} has 22\% sampled stored instructions that trigger cache invalidation, while the default Linux allocator only has 3\% of such instructions.
%The result proves that \MP{} could detect whether a program is suffering from severe false sharing issues introduced by the allocator. 

%In \texttt{fluidanimate} running with \texttt{DieHarder}, \MP{} indicates its page utilization is only 79\%, while the default Linux allcaor is 97\%. That abnormal number could indicate \texttt{DieHarder} does not fit well with the memory usage pattern and the access pattern of \texttt{fluidanimate}. Actually, \texttt{fluidanimate} that runs with \texttt{DieHarder} has 25\% higher memory overhead, and has 25\% more page faults than the default Linux allocator. 
%All the aspects consistently show that \texttt{DieHarder} does not tap well with \texttt{fluidanimate}.

For \texttt{kmeans}, both \texttt{glibc-2.28} (the default one) and \texttt{glibc-2.21} runs more than  10\% slower than other allocators. \MP{} reports that their page and cache utilization are around 66\% and 65\% separately. But the page and cache utilization rate for other allocators are about 85\% and 87\%, which explains the slowness of using \texttt{glibc-2.28}  and \texttt{glibc-2.21}. This is also the reason why \MP{} cannot predict such performance impact in Table~\ref{tbl:predictionResult}, since \MP{} can only predict the impact caused by slow memory management operations. 

%In fact, those two allocators run around 10\% slower than other allocators in \texttt{kmeans}, though their time for allocations have almost no difference with others. Thus, \MP{}'s information of utilization provides a useful clue of the slowdown that \texttt{kmeans} does not fit well with the default Linux allocator and \texttt{glibc-2.21}.



\subsubsection{Benefiting Allocator Designers}
\label{sec: benifitdesigners}
As described before, \MP{} will benefit allocator designers by presenting more details related to the performance and memory overhead. All types of data can be utilized in the following order. First, we could check whether the prediction reports a potential performance improvement. If not, we should check whether passive/active false sharing should be avoided, or page/cache utilization rate can be reduced or not. Otherwise, some performance improvement indicates the implementation issue for the current allocator. The synchronization issue should be checked and fixed first, if there are some reported lock issues. Then we could check whether the cycles for each memory management operation, the reported instruction numbers of allocations/deallocations, the hardware events, and then the runtime and number of memory-related system calls. Due to the space limit, we only utilize multiple examples to show the effectiveness and helpfulness of \MP{}. 
%small/middle/big size of objects. If the numbers are in the suggested range (as discussed above), then we could check the next item. Otherwise, we may need to improve the implementations to reduce the number of instructions. Fourth, we will check whether the cycles of allocations/deallocations are in the suggested range. If the number of instructions is in the suggested range, but the cycles are not. Then we should check whether the issue is caused by the user-space or kernel-space synchronization. Fifth, we will check whether mmprof reports the lock contention in which type of allocation/deallocation (user-space synchronization). More specifically, mmprof also reports the number of lock acquisitions and the contending number for each lock. Sixth, we will check whether mmprof reports the potential kernel-space contention by checking the cycles of each memory-related system call.


%For performance, it shows the number of instructions, page faults and cache misses for each operation, the potential issues caused by synchronizations and system calls, different application-friendliness metrics. For memory overhead, it will show different types of memory overhead. 




%\todo{What are workflow?}


\begin{table}[h]
  \centering
  \footnotesize
  \setlength{\tabcolsep}{0.2em}
\begin{tabular}{l | l | l | l | l}
\hline
Applications & Allocator & Abnormal Metrics & Possible Root Cause \\ \hline
cache-thrash & TcMalloc & $47.7\times$ slowdown & Contention rate for PFS lines: 50\% & Root Cause \RN{1} \\ \hline
dedup & glibc-2.21 & 20\% slowdown &  \# of Madvise for small allocations: & Root Cause \RN{2} \\ \hline
freqmine & jemalloc &  Memory consumption & memory blowup: 2174230K (37\%) & Root Cause \RN{3} \\ 
 & &  & external fragmentation: 1132045K (19\%) \\ \hline
swaptions  &  DieHarder  & $9\times$ slowdown  & 
	Small reused alloc: 377476 cycles 	& Root Cause \RN{4} \\
& & & Small free: 331745 cycles, and 4.9 cache misses & \\ 
& & & Per-lock acquisition: 353448 cycles & \\\cline{4-5}
& & & External fragmentation: 1878K(37\%) & Root Cause \RN{5} \\
& & & cache utilization 55\%, page utilization 35\% & \\\cline{2-5}
& Hoard & $6.3\times$ slowdown & 
	Small reused alloc: 68933 cycles, 9.4 cache misses 	& Root Cause \RN{6} \\
	
& & & Small free: 53402 cycles, 11.8 cache misses & \\ 
& & & 1.54 locks per-operation & \\
\cline{4-5}
& & & Memory blowup: 4789K( 81\%) & Root Cause \RN{7} \\
& & & cache utilization 62\%, page utilization 51\% & \\\cline{2-5}
& OpenBSD & $8\times$ slowdown & Small re-used alloc: 98962 cycles, 4.5 cache misses & Root Cause \RN{8}\\ 
& & & Small free: 101081 cycles, 7 cache misses & \\ 
& & & 1.1 locks per-operation &  \\ \hline


%\multirow{2}{*}{Performance} & {Alloc/Free runtime} & Timestamp\\ \cline{2-3}

  \end{tabular}
  \centering
  \caption{Abnormal metrics of allocators for different applications.\label{table:abnormal}}
\end{table}

Due to the space limit, we only select multiple examples with abnormal metrics of allocators for the analysis. We aim to cover all allocators, with their abnormal data as shown in Table~\ref{table:abnormal}. Based on these listed metrics, we will show the helpful guidelines provided by \MP{} when analyzing the performance and memory issue. In the end of this section, we also provide some observations based on the evaluation of these allocators.  

 \paragraph{TcMalloc:}
\texttt{TcMalloc} typically performs very well in almost all applications, except for few synthetic applications, such as \texttt{cache-thrash}, \texttt{cache-scratch}, and \texttt{threadtest}. 

\textit{Root Cause \RN{1}}:
For \texttt{cache-thrash}, it runs around $47.7\times$ slower compared to the default Linux allocator. Using \MP{}, we find that the runtime of allocations and deallocations of TcMalloc is actually at a normal range. The only obvious issue is that it has around a 50\% cache contention rate for cache lines with passive false sharing issues, which is the major reason causing the significant slowdown. By checking the source code, we observe that TcMalloc will actually experience both active and passive false sharing issues. For active false sharing, TcMalloc will get one object for a thread from its central heap, so that two continuous objects can be utilized by two different threads. Since TcMalloc always places a freed object to the current thread's per-thread cache, which will also introduce a passive false sharing issue. In comparison, the Linux allocator always returns an object back to its original owner, avoiding passive false sharing.  

\paragraph{glibc-2.21:}
\textit{Root Cause \RN{2}}: The allocator of glibc-2.21 has a bug that invokes excessively large number of \texttt{madvise} systems calls under certain memory use patterns~\cite{madvise}, which is exhibited clearly when running the dedup application. \MP{} reports around 31218 invocations of \texttt{madvise} per second (with a total of 505773 in 16.2 seconds), and the runtime of each \texttt{madvise} is about $23598$ cycles that is $10\times$ of the normal runtime. This clearly indicates that too many \texttt{madvise} system calls introduce contention inside the kernel. Changing the threshold of \texttt{madvise} improves the performance by 20\%.

\paragraph{jemalloc:} jemalloc typically has good performance, but has greater memory consumption.

\textit{Root Cause \RN{3}}: For \texttt{freqmine} application, jemalloc utilizes 6\% more memory than the default Linux allocator, and 36\% more than TcMalloc. Via the report, we can know that jemalloc introduces around 37\% memory blowup and 19\% of external fragmentation of its total memory consumption. In comparison, TcMalloc only has 1\% memory blowup and 13\% external fragmentation.  

\paragraph{DieHarder:} DieHarder performs much slower than other allocators for many applications, and runs $9\times$ slower than the default Linux allocator for \texttt{swaptions}.

\textit{Root Cause \RN{4}}:Based on evaluation results in Table~\ref{table:abnormal}, DieHarder has multiple design issues. From the runtime and lock-related information, we can determine that this allocator introduces an abnormally high amount of cache misses (4.9) for each deallocation. By examining the code, DieHarder must check all miniheaps to identify whether an object belongs to a particular miniheap. This design is not only very slow, but also introduce multiple cache misses by its search. Also, DieHarder utilizes a central lock for all allocations and deallocations, with four locks in total. This design will introduce large slowdowns for parallel applications, which explains why each lock acquisition will take $353,448$ cycles. 

\texttt{Root Cause \RN{5}}: we also notice that DieHarder introduces external fragmentation, around 37\%. As described before, this also includes the size of skipped objects, which is caused by DieHarder's over-provision allocation mechanism. Since DieHarder will also randomly choose some objects, that is maybe the cause of its low cache utilization and page utilization. 


\paragraph{Hoard:} 
 \texttt{Hoard} is running around $6.3\times$ slower than the default allocator. Based on our analysis, it can be caused by multiple reasons.
 
 \texttt{Root Cause \RN{6}:}
 The output of \MP{} shows that it has a large runtime for each allocation and deallocation, and has around 11.8 cache misses. Also, \MP{} reports that it has 1.54 lock acquisitions per call. Clearly, Hoard has a big issue of using locks. By checking the code, we found that Hoard at least acquires a lock for each allocation and deallocation, which is $95713\times$ more than of locks of TcMalloc. TcMalloc utilizes a per-thread cache that there is no need to acquire the lock if an allocation can be satisfied from the per-thread cache. Instead, by using too many locks, Hoard will introduce more cache misses unnecessarily. Another issue is that Hoard are using so many instructions due to its deep-level of templates. For instance, its per-deallocation will has around 1322 instructions, while TcMalloc only has 73.7 instructions and 222 cycles.  
 
 \texttt{Root Cause \RN{7}:} Hoard also has a big issue of memory blowup, with 81\% memory blowup for \texttt{swaptions}. Also, it also much lower cache utilization and page utilization rate than TcMalloc, where TcMalloc's cache and page utilization rate is 80\% and 73\%. That is, all of these factors of Hoard will contribute to the slowdown on this application.
 
\paragraph{OpenBSD:} \texttt{Root Cause \RN{8}:}  OpenBSD has $8\times$ slowdown for \texttt{swaptions}, comparing to the default allocator. Based on its report, we find out that OpenBSD has the similar issue as Hoard, since it acquires more than one lock for each operation. By checking the code, we find out that OpenBSD has the same global lock for all allocations and deallocations, which is the possible reason for its big slowdown. Also, OpenBSD also has a big cache misses for its re-used allocations and deallocations for small objects, which is possibly another reason why it has a big slowdown. For OpenBSD, we also observe that it has significant big number of instructions than other allocators, which as 430 instructions for deallocating a small object, and 295 instructions for a re-used allocation. This is possibly another reason for its big slowdown. 



%\paragraph{jemalloc:}
%During evaluation, the \texttt{reverse\_index} benchmark was found to perform approximately 21\% slower when paired with \texttt{jemalloc} versus the default Linux allocator. Upon inspection, we find that, with \texttt{jemalloc}, the program exhibited over $2x$ the number of CPU cycles associated with the deallocation execution path, as well as a 34\% increase in critical section duration (i.e., the cycles spent within outermost critical sections).




 
\begin{comment}
\renewcommand{\arraystretch}{1.5}
\begin{table}[!ht]
  \centering
   \caption{Important   Metrics\label{tab:metrics}}
  
    \begin{tabular}{l|l|l|l}
    \hline
\multirow{5}{*} {Performance} & \multirow{3}{*}{Allocation Runtime} & New Allocation  (Small) & 80\\ \cline{3-4}
& & Reallocation  (Small) & 1000 \\ \cline{3-4}
& &  Large Allocation & 1000 \\ \cline{2-4}
& \multirow{2}{*}{Deallocation Runtime} & Small  &  \\ \cline{3-4}
& & Large & 100 \\ \cline{1-4}
    
    \end{tabular}
\end{table}
	
\end{comment}




%For a performant allocator, what's the common things within the average allocator. We could utilize a table to list the average points of each allocator. Potentially, we could utilize these parameters to evaluate a new allocator. 

%For evaluating purpose, we could provide two information, one is the average with all evaluated allocators, another one is to omit one allocator with the lowest scores. 


%It seems that BIBOP style allocators are the trend of allocators, which not only has a better performance overhead on average, but also has better safety by separating the metadata from the actual heap. 


