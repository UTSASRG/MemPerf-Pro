\section{Overview}

\MP{} is designed as a drop-in library that can be simply linked to applications (and allocators), without the requirement of changing the code or even re-compilation. Although \MP{} could employ the binary-instrumentation to perform a more detailed profiling, such as identifying the instructions within each allocation or deallocation, the binary instrumentation may impose prohibitive performance overhead. With a high overhead, the profiling results may be significantly skewed, such as the waiting time of each lock acquisition inside the allocation. Instead, \MP{} employs the PMU hardware, RDTSC timestamp hardware, and simple counters together to perform the profiling.  
 
As described above, \MP{} focuses on two aspects of the allocator, the allocator itself and its potential impact on applications (application friendliness). In order to adapt to different allocators, a small program is designed to collect the allocator's specific feature, such as the type of allocators, the size class information, or others. 

In the remainder of this section, we describe the profiled items for each category, and the basic idea of profiling. 

\subsection{Background of Allocators}

\label{sec:allocator}

There exists multiple types of allocators, such as bump-pointer, BIBOP, and region-based allocators. However, region-based allocators are only suitable for special situations that all allocated objects within the same region can be efficiently deallocated at once~\cite{Gay:1998:MME:277650.277748}, which is not the common type for popular allocators. Based on the common knowledge,  the number of requests for small objects is significantly much larger than the number of big objects. Thus, most allocators utilize different mechanisms to manage ``small'' and ``big'' objects, where the bump-pointer and BIBOP style typically indicates their mechanisms to manage small objects. For big objects, allocators may map a block of memory from the OS directly during an allocation, and then return it to the OS via \texttt{munmap}~\cite{Hoard}.   

\textbf{Bump-pointer} allocators typically utilize a  arithmetic operation to bump a pointer when the block is first allocated~\cite{Cling}, in which objects with different sizes can be allocated adjacently. Freed objects are typically placed into different free lists, based on their size classes.  Therefore, they typically have to get the size information of each object upon deallocations. Since objects with different size classes are placed continuously, it will be slow to look up the size information if placed in other placements (no way to compute). Therefore, the size information is typically placed on the header of objects to support the fast look up.  Two examples of this type are the Glibc allocator, the default allocator in Linux that originates from dlmalloc~\cite{dlmalloc}, and Hoard~\cite{Hoard}.  

The \textbf{BIBOP} allocators belong to another common type of allocators, where the BIBOP stands for '' Big Bag of Pages''~\cite{hanson1980}. For these allocators, one or multiple continuous pages are treated as a ``bag'', holding objects with the same size class. The metadata of heap objects, such as its size and availability information, is typically stored in a separate area. Thus, BIBOP-style allocators improve the security and reliability, by avoiding metadata corruption caused by buffer overflows. Many performance-oriented allocators, such as TCMalloc~\cite{TCMalloc}, \texttt{jemalloc}~\cite{jemalloc}, and most secure allocators, such as OpenBSD~\cite{OpenBSD} and DieHarder~\cite{DieHarder}, belong to this type. BIOBP allocators may utilize free lists or bitmaps to manage the availability of objects. When using the bitmap, only one bit metadata is sufficient to track the availability of an object, which may impose less memory overhead for the metadata but with possibly higher performance overhead.  


Almost all existing allocators manage objects by  size classes, instead of the exact size of objects. But different allocators support  different size classes: some only support power-of-2 sizes, such as DieHarder~\cite{DieHarder} or the OpenBSD allocator~\cite{OpenBSD}, while some utilize more fine-grained size classes, such as the Glibc allocator and jemalloc~\cite{jemalloc}. For each allocation request, the allocator will round up the requested size to the nearest size class. Upon each deallocation, the deallocated object will be threaded into the corresponding free list (for the freelist design) or the corresponded bit will be cleared (for the bitmap design). Therefore, the idea of size class enables the re-use of heap objects. However, it may introduce \textit{internal fragmentation}, wasting the space between the actual allocated size and the size class. 

In order to support multithreaded applications, most allocators typically support per-processor heap design, one idea introduced by Hoard~\cite{Hoard}. For instance, Hoard maintains per-processor heaps and a global heap concurrently. Allocation requests from different threads typically will be satisfied from different per-processor heaps, avoiding false sharing and reducing the lock contention at the same time. Thus, this design actually augments the scalability of allocators. However, it may introduce the \textit{memory blowup}, one typical issue existing in multithreaded allocators. Based on our understanding and experiments, the memory blowup is the one major reason why different allocators may impose different physical memory consumption. 

\subsection{Basic Idea of maprof}
\MP{} profiles different types of data as discussed in the following. 
  
\subsubsection{Performance Overhead}

For performance overhead, \MP{} profiles the average data of each allocation and deallocation, instead of the summarized values over the whole execution. Per allocation/deallocation data helps identify internal issues inside, if the data is counterintuitive. For instance, \texttt{DieHarder} is identified to have a very high number (\todo{up to X}) of cache misses upon each deallocation, which clearly a deficiency. 

First, \MP{} collects the time spending on each allocation and deallocation. It intercepts allocations and deallocations of allocations, and then computes the RDTSC timestamp difference of before and after each operation as the execution time. The allocation and deallocation time is a very important metrics on the efficiency of an allocator.  

%TLB read misses/TLB write misses/page faults/cache misses/instructions. They are PMU-based. 
% 

Second, \MP{} further collects hardware PMU events for each allocation and deallocation, where the introduction of PMU is described in Section~\ref{sec:pmu}. The PMU events allow \MP{} discover the specific information of allocations/deallocations without the instrumentation, such as retired instructions, cache misses, and TLB read/write misses. 

%Based on our understanding, the hardware events, such as retired instructions, cache misses or TLB misses, will help reveal some implementation issues of an allocator. For instance, \MP{} detects that the DieHarder allocator has an excessive number of cache misses and TLB misses upon each deallocation, around 5 times of each deallocation, which is significantly larger than that of other allocators. By examining the code, we found out a serious implementation issue of the DieHarder allocator, which traverses all mini-heaps to identify the placement of each object. Obviously, this implementation is extremely slow, considering that every deallocation has to perform such expensive lookup. By fixing such issues, the DieHarder's performance improves around \todo{20\%}.     


\begin{comment}
Can we integrate the cache misses or page faults for each allocation and deallocation, so that we could identify the issue of DieHarder that invokes many unnecessary cache misses?

If we could correlate cache misses to each thread, then we could do this. 

If allocation and deallocation takes too much time, it could be caused by multiple reasons:

(1) First, it just takes a lot of instructions (could we find out the lapsed instructions for each thread?)
(2) It may be caused by not good algorithm? 
(3) It can be caused by lock contention?
(4) It can be caused by system call related contention?
\end{comment} 

\subsubsection{Memory Overhead}

Based on our understanding, the memory overhead of each allocator comes from three aspects: the metadata overhead, the alignment (or internal fragmentation), and the memory blowup. 
How much memory are due to memory re-utilization rate? For instance, we may not fully utilize the memory due to the randomization mechanism. We only care about physical memory waste. Then we should also investigate how much memory has been explicitly returned back to the OS. 

How much memory are due to the metadata itself? 

\subsubsection{Scalability Analysis} 

User space contention:
How many separate locks are explicitly utilized? 
How many lock acquisitions? How much time are spending on lock waiting for each thread, and in total?

How much time spending on kernel-space contention? For instance, we could infer from memory-related system calls, such as mmap, munmap, madvise, brk, or something else? 

That is, we may have to integrate with SyncPerf for doing this. We will borrow their implementation in order to do this. 

\subsubsection{Application Friendliness} 
How many page faults and cache misses that are caused by applications? 

How many remote accesses? How many interconnect messages? We may employ the PMU mechanism to identify the information.

\subsection{Technical Challenges}

As described above, \MP{} employs hardware PMU, RDTSC, and simple counters together to perform the profiling. Also, as described above, there are some challenges. 
 

How to support the default Linux allocator, since it actually utilizes some internal function calls, which cannot be intercepted by \MP{} normally. Therefore, we actually migrated its allocator as a separate library, as other allocators. This allows \MP{} to intercept all lock acquisitions, system calls as usual. 

How to track the size information of objects for both bump-pointer allocators and BIBOP allocators? We observe that it is difficult to track the range of objects. In fact, we also utilize the ?

How to know the metadata overhead of each allocator? It is very difficult to identify which region is used to save the allocator information, without changing the source code of the allocator. Also, it is even difficult to know. We observe that the allocator overhead comes from multiple aspects: metadata overhead, alignment overhead (or internal fragmentation introduced by size classes), and then memory blowup. 
As far as we know the total memory overhead, then the metadata overhead = total heap size - alignment overhead - memory blowup overhead - memory usage.

However, if the allocator itself utilizes the global variable or an pre-allocated block as the metadata, which can be omitted by \MP{}. 

 


\subsection{Background of Performance Monitoring Units}
\label{sec:pmu}

\subsection{Background of RDTSC}

\label{sec:rdtsc}






