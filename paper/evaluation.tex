\section{Experimental Evaluation}
\label{sec:evaluation}

We will investigate the following allocators. We choose two general allocators, e.g. TCMalloc and TCMalloc, and two secure allocators, i.e. DieHarder and Guarder, in addition to the default allocator of the Linux system. Secure allocators will employ the randomization for memory allocations, which . 
For the default allocator, we actually have to 

\subsection{Performance Overhead}

We will evaluate the performance overhead of the profiler itself. 

\subsection{Memory Overhead}
We will evaluate the performance overhead of the profiler itself. 

\subsection{Comparison of Different Allocators}

\subsubsection{Performance Overhead}

\subsubsection{Memory Overhead}

\subsubsection{Scalability}

\subsubsection{Application Friendliness}

\subsubsection{Conclusions}

For a performant allocator, what's the common things within the average allocator. We could utilize a table to list the average points of each allocator. Potentially, we could utilize these parameters to evaluate a new allocator. 

For evaluating purpose, we could provide two information, one is the average with all evaluated allocators, another one is to omit one allocator with the lowest scores. 


It seems that BIBOP style allocators are the trend of allocators, which not only has a better performance overhead on average, but also has better safety by separating the metedata from the actual heap. 

\subsection{Issues Identified in Different Allocators}

\paragraph{Glibc-2.XXX:}

\paragraph{DieHarder:}

\paragraph{jemalloc:}

\paragraph{TCMalloc:}

