\section{Experimental Evaluation}
\label{sec:evaluation}

The experimental evaluation will answer the following questions:
\begin{itemize}
\item How is the effectiveness of \MP{}? (Section~\ref{sec:effectiveness}) 	
\item What is the performance overhead of \MP{}? (Section~\ref{sec:perf})
\item What is the memory overhead of \MP{}? (Section~\ref{sec:memory})
\end{itemize}

Experiments were performed on a two-processor machine, where each processor is Intel(R) Xeon(R) Gold 6230. Each processor has 20 cores in total. This machine has 256GB of main memory, 20MB of L2 cache, and 1280KB L1 cache. The underlying OS is Ubuntu 18.04.3 LTS, installed with the Linux-5.3.0-40. All applications were compiled using GCC-7.5.0, with -O2 and -g flags.

\subsection{Effectiveness}
\label{sec:effectiveness}

In order to evaluate the effectiveness, we evaluate \MP{} with five general purpose allocators, e.g. two versions of the Linux allocator (version 2.21 and 2.28), TCMalloc~\citep{tcmalloc}, jemalloc, and Hoard, and two secure allocators, i.e. DieHarder and OpenBSD. These allocators include both sequential and BiBOP-style allocators. Secure allocators were included, since they have their unique memory management policies. 

For the evaluation, we use the default configurations of these allocators. However, we make some changes in order to the interception of synchronizations. Since Linux allocators are included in \texttt{glibc} libraries, they invokes the internal synchronizations as \texttt{lll\_lock}, which cannot be intercepted by \MP{}. They are compiled separately as a stand-alone library. Since Hoard are using \texttt{std::lock\_guard} for its synchronization, which cannot be intercepted, we replaced them with POSIX spin locks to track its synchronization behavior.

%\todo{Let's use a table to list all dramatic difference between these allocators. This gives us some evidence of allocators}
%\subsection{Issues Identified in Different Allocators}


\begin{table}[h]
  \centering
  \footnotesize
  \setlength{\tabcolsep}{0.2em}
\begin{tabular}{l | l | l | l | l}
\hline
Applications & Allocator & Abnormal Metrics & Possible Root Cause \\ \hline
cache-thrash & TcMalloc & $47.7\times$ slowdown & Contention rate for PFS lines: 50\% & Root Cause \RN{1} \\ \hline
dedup & glibc-2.21 & 20\% slowdown &  \# of Madvise for small allocations: & Root Cause \RN{2} \\ \hline
freqmine & jemalloc &  Memory consumption & memory blowup: 2174230K (37\%) & Root Cause \RN{3} \\ 
 & &  & external fragmentation: 1132045K (19\%) \\ \hline
swaptions  &  DieHarder  & $9\times$ slowdown  & 
	Small reused alloc: 377476 cycles 	& Root Cause \RN{4} \\
& & & Small free: 331745 cycles, and 4.9 cache misses & \\ 
& & & Per-lock acquisition: 353448 cycles & \\\cline{4-5}
& & & External fragmentation: 1878K(37\%) & Root Cause \RN{5} \\
& & & cache utilization 55\%, page utilization 35\% & \\\cline{2-5}
& Hoard & $6.3\times$ slowdown & 
	Small reused alloc: 68933 cycles, 9.4 cache misses 	& Root Cause \RN{6} \\
	
& & & Small free: 53402 cycles, 11.8 cache misses & \\ 
& & & 1.54 locks per-operation & \\
\cline{4-5}
& & & Memory blowup: 4789K( 81\%) & Root Cause \RN{7} \\
& & & cache utilization 62\%, page utilization 51\% & \\\cline{2-5}
& OpenBSD & $8\times$ slowdown & Small re-used alloc: 98962 cycles, 4.5 cache misses & Root Cause \RN{8}\\ 
& & & Small free: 101081 cycles, 7 cache misses & \\ 
& & & 1.1 locks per-operation &  \\ \hline


%\multirow{2}{*}{Performance} & {Alloc/Free runtime} & Timestamp\\ \cline{2-3}

  \end{tabular}
  \centering
  \caption{Abnormal metrics of allocators for different applications.\label{table:abnormal}}
\end{table}

Due to the space limit, we only select multiple examples with abnormal metrics of allocators for the analysis. We aim to cover all allocators, with their abnormal data as shown in Table~\ref{table:abnormal}. Based on these listed metrics, we will show the helpful guidelines provided by \MP{} when analyzing the performance and memory issue. In the end of this section, we also provide some observations based on the evaluation of these allocators.  

 \paragraph{TcMalloc:}
\texttt{TcMalloc} typically performs very well in almost all applications, except for few synthetic applications, such as \texttt{cache-thrash}, \texttt{cache-scratch}, and \texttt{threadtest}. 

\textit{Root Cause \RN{1}}:
For \texttt{cache-thrash}, it runs around $47.7\times$ slower compared to the default Linux allocator. Using \MP{}, we find that the runtime of allocations and deallocations of TcMalloc is actually at a normal range. The only obvious issue is that it has around a 50\% cache contention rate for cache lines with passive false sharing issues, which is the major reason causing the significant slowdown. By checking the source code, we observe that TcMalloc will actually experience both active and passive false sharing issues. For active false sharing, TcMalloc will get one object for a thread from its central heap, so that two continuous objects can be utilized by two different threads. Since TcMalloc always places a freed object to the current thread's per-thread cache, which will also introduce a passive false sharing issue. In comparison, the Linux allocator always returns an object back to its original owner, avoiding passive false sharing.  

\paragraph{glibc-2.21:}
\textit{Root Cause \RN{2}}: The allocator of glibc-2.21 has a bug that invokes excessively large number of \texttt{madvise} systems calls under certain memory use patterns~\cite{madvise}, which is exhibited clearly when running the dedup application. \MP{} reports around 31218 invocations of \texttt{madvise} per second (with a total of 505773 in 16.2 seconds), and the runtime of each \texttt{madvise} is about $23598$ cycles that is $10\times$ of the normal runtime. This clearly indicates that too many \texttt{madvise} system calls introduce contention inside the kernel. Changing the threshold of \texttt{madvise} improves the performance by 20\%.

\paragraph{jemalloc:} jemalloc typically has good performance, but has greater memory consumption.

\textit{Root Cause \RN{3}}: For \texttt{freqmine} application, jemalloc utilizes 6\% more memory than the default Linux allocator, and 36\% more than TcMalloc. Via the report, we can know that jemalloc introduces around 37\% memory blowup and 19\% of external fragmentation of its total memory consumption. In comparison, TcMalloc only has 1\% memory blowup and 13\% external fragmentation.  

\paragraph{DieHarder:} DieHarder performs much slower than other allocators for many applications, and runs $9\times$ slower than the default Linux allocator for \texttt{swaptions}.

\textit{Root Cause \RN{4}}:Based on evaluation results in Table~\ref{table:abnormal}, DieHarder has multiple design issues. From the runtime and lock-related information, we can determine that this allocator introduces an abnormally high amount of cache misses (4.9) for each deallocation. By examining the code, DieHarder must check all miniheaps to identify whether an object belongs to a particular miniheap. This design is not only very slow, but also introduce multiple cache misses by its search. Also, DieHarder utilizes a central lock for all allocations and deallocations, with four locks in total. This design will introduce large slowdowns for parallel applications, which explains why each lock acquisition will take $353,448$ cycles. 

\texttt{Root Cause \RN{5}}: we also notice that DieHarder introduces external fragmentation, around 37\%. As described before, this also includes the size of skipped objects, which is caused by DieHarder's over-provision allocation mechanism. Since DieHarder will also randomly choose some objects, that is maybe the cause of its low cache utilization and page utilization. 


\paragraph{Hoard:} 
 \texttt{Hoard} is running around $6.3\times$ slower than the default allocator. Based on our analysis, it can be caused by multiple reasons.
 
 \texttt{Root Cause \RN{6}:}
 The output of \MP{} shows that it has a large runtime for each allocation and deallocation, and has around 11.8 cache misses. Also, \MP{} reports that it has 1.54 lock acquisitions per call. Clearly, Hoard has a big issue of using locks. By checking the code, we found that Hoard at least acquires a lock for each allocation and deallocation, which is $95713\times$ more than of locks of TcMalloc. TcMalloc utilizes a per-thread cache that there is no need to acquire the lock if an allocation can be satisfied from the per-thread cache. Instead, by using too many locks, Hoard will introduce more cache misses unnecessarily. Another issue is that Hoard are using so many instructions due to its deep-level of templates. For instance, its per-deallocation will has around 1322 instructions, while TcMalloc only has 73.7 instructions and 222 cycles.  
 
 \texttt{Root Cause \RN{7}:} Hoard also has a big issue of memory blowup, with 81\% memory blowup for \texttt{swaptions}. Also, it also much lower cache utilization and page utilization rate than TcMalloc, where TcMalloc's cache and page utilization rate is 80\% and 73\%. That is, all of these factors of Hoard will contribute to the slowdown on this application.
 
\paragraph{OpenBSD:} \texttt{Root Cause \RN{8}:}  OpenBSD has $8\times$ slowdown for \texttt{swaptions}, comparing to the default allocator. Based on its report, we find out that OpenBSD has the similar issue as Hoard, since it acquires more than one lock for each operation. By checking the code, we find out that OpenBSD has the same global lock for all allocations and deallocations, which is the possible reason for its big slowdown. Also, OpenBSD also has a big cache misses for its re-used allocations and deallocations for small objects, which is possibly another reason why it has a big slowdown. For OpenBSD, we also observe that it has significant big number of instructions than other allocators, which as 430 instructions for deallocating a small object, and 295 instructions for a re-used allocation. This is possibly another reason for its big slowdown. 



%\paragraph{jemalloc:}
%During evaluation, the \texttt{reverse\_index} benchmark was found to perform approximately 21\% slower when paired with \texttt{jemalloc} versus the default Linux allocator. Upon inspection, we find that, with \texttt{jemalloc}, the program exhibited over $2x$ the number of CPU cycles associated with the deallocation execution path, as well as a 34\% increase in critical section duration (i.e., the cycles spent within outermost critical sections).




 
\begin{comment}
\renewcommand{\arraystretch}{1.5}
\begin{table}[!ht]
  \centering
   \caption{Important   Metrics\label{tab:metrics}}
  
    \begin{tabular}{l|l|l|l}
    \hline
\multirow{5}{*} {Performance} & \multirow{3}{*}{Allocation Runtime} & New Allocation  (Small) & 80\\ \cline{3-4}
& & Reallocation  (Small) & 1000 \\ \cline{3-4}
& &  Large Allocation & 1000 \\ \cline{2-4}
& \multirow{2}{*}{Deallocation Runtime} & Small  &  \\ \cline{3-4}
& & Large & 100 \\ \cline{1-4}
    
    \end{tabular}
\end{table}
	
\end{comment}




%For a performant allocator, what's the common things within the average allocator. We could utilize a table to list the average points of each allocator. Potentially, we could utilize these parameters to evaluate a new allocator. 

%For evaluating purpose, we could provide two information, one is the average with all evaluated allocators, another one is to omit one allocator with the lowest scores. 


%It seems that BIBOP style allocators are the trend of allocators, which not only has a better performance overhead on average, but also has better safety by separating the metadata from the actual heap. 



\subsection{Performance Overhead}
\label{sec:perf}

\begin{figure}[!ht]
\centering
\includegraphics[width=5.5in]{figures/perfoverhead}
\caption{Performance overhead of \texttt{mmprof}, normalized to the runtime of default Linux allocator.\label{fig:overhead}}
\end{figure}

We also evaluate the performance overhead of 
\MP{} using PARSEC~\citep{parsec},  Phoenix~\citep{phoenix}, and multiple synthetic applications from Hoard~\cite{Hoard}. The performance overhead can be seen in Figure~\ref{fig:overhead}, which is the average value of five executions. From this figure, we can see that \MP{} runs $2.6\times$ slower than the default allocator, where two applications (\texttt{histogram} and \texttt{threadtest}) impose over $5\times$ performance overhead. Based on our understanding, both the number of memory operations and the number of lock acquisitions can significant impact  the performance overhead of \MM{}. To help the explanation, we further collect the characteristics of these applications, as shown in Table~\ref{table:characteristics}.  

First, if an application invokes extensive applications and deallocations in a short period of time, then \MP{} may introduce a large performance overhead. For each memory operation, \MP{} invokes two RDTSC instructions to collect the runtime,  updates multiple counters, and updates the state in the global hash table.  Second, the number of lock acquisitions inside memory operations could also significantly affect the performance. Similarly, \MP{} also collects the runtime data of each lock acquisition via the RDTSC instruction, and update different counters. 
Since \texttt{canneal} invokes 11 million memory operations and 0.3 million locks acquisitions per second, and \texttt{threadtest} will has 6 million memory operations and 33 lock acquisitions per second, that explains why these applications have larger performance overhead with \MP{}.  

\texttt{histogram} program is an exception, since it has a small number of allocations. For this application, \MP{} has an initialization phase to perform the initialization, and an finalization phase to analyze the data and write out results to the external file. But \texttt{histogram} only takes 0.12 seconds to finish. Therefore, \MP{} adds more overhead than the program's execution time. \texttt{linear\_regression} has the same issue, with the total execution time of 0.3 seconds.  
%for some tiny applications like \texttt{histogram}, which only takes 0.1 second when running along, \MP{}'s initialization and conclusion would take more time than themselves. Thus, performance overhead ratios for tiny applications could be larger.
%\texttt{threadtest} actually imposes more performance overhead, due to the reason that most memory opeations are actually 



%some applications invoke allocations intensively and almost simultaneously from different threads, \MP{} may introduce more contention in the hash table when checking every object's status, and higher contention rates would cause programs' slowdown.
\begin{comment}
For example, according to \ref{sec:memory}, 
canneal 2.86x   11438194.73 (alloc+free)/sec    42282925 alloc+free 311044.69 lockacqs/sec
reverse_index 4.59x 973245.43 (alloc+free)/sec 40000387 alloc+free 120407.33 lockacqs/sec
threadtest 7.21x 6 228 715.40 (alloc+free)/sec 256000203 alloc+free 33 239 532.98 lockacqs/sec



For example, according to \ref{sec:memory}, 
linear_regression 4.56x, 0.3s when running alone, 2 alloc+free
word_count 4.26x, 1.71s when running alone, 1 alloc
histogram 10.23x, 0.12s when running alone, 4 alloc+free

Upon every allocation and deallocation, \MP{} collects the runtime and acquisition information.

\end{comment}

\begin{table}[h]
  \centering
  \footnotesize
  \setlength{\tabcolsep}{0.2em}
\begin{tabular}{l|c|r|r|r|r}
\hline
\multicolumn{1}{c|}{Application} & 
\multicolumn{1}{c|}{Runtime}    & 
\multicolumn{1}{c|}{New Alloc}     & 
\multicolumn{1}{c|}{Reused Alloc}     & 
\multicolumn{1} {c|}{Free}     & 
\multicolumn{1}{c}{Lock Acqs} \\ \hline
  blackscholes & 16.7 & 8 & 1 & 7 & 11 \\ \hline   
   bodytrack & 8.5 & 20150 & 460616 & 480765 & 871397 \\ \hline    
   cache-scratch & 3.0 & 44 & 400000 & 400043 & 47 \\ \hline    
   cache-thrash  & 2.4 & 43 & 3999960 & 4000002 & 45\\ \hline  
   canneal & 29.4 & 8756242 & 12385221 & 21141462 & 9144714 \\ \hline    
   dedup & 12.7 & 3384984 & 683368 & 1750378 & 4864027 \\ \hline    
   facesim & 159.2 & 953143 & 3955049 & 4094483 & 1678963 \\ \hline    
   ferret & 25.3 & 149680 & 236867 & 415914 & 417370\\ \hline    
   fluidanimate & 12.3 & 229912 & 1 & 229913 & 307124 \\ \hline    
   freqmine & 20.2 & 1810 & 4 & 1070 & 15926 \\ \hline    
   histogram & 0.12 & 2 & 0 & 2 & 3 \\ \hline    
   kmeans & 16.4 & 200691 & 533 & 200579 & 303705 \\ \hline    
   larson & 15.1 & 2408955 & 33726797 & 36095750 & 38088835 \\ \hline   
   linear_regression & 0.3 & 1 & 0 & 1 & 2 \\ \hline    
   matrix_multiply & 4.8 & 83 & 0 & 82 & 85 \\ \hline    
   pca & 9.2 & 16131 & 29 & 72 & 16466 \\ \hline    
   raytrace & 41.1 & 5000115 & 15000100 & 20000172 & 5000240 \\ \hline   
   reverse_index & 1.5 & 1632810 & 106173 & 1738982 & 1806110\\ \hline  
   streamcluster & 23.5 & 47 & 8798 & 8844 & 17622\\ \hline    
   string_match & 0.6 & 8 & 0 & 7 & 10 \\ \hline    
   swaptions & 14.5 & 2040 & 47999756 & 48000385 & 48002039\\ \hline    
   threadtest & 7.7 & 1280122 & 126720000 & 128000081 & 255944404\\ \hline    
   vips 6.5 & 8128 & 1420072 & 1428019 & 1526404\\ \hline    
   word_count & 1.7 & 1 & 0 & 0 & 2\\ \hline   
   x264 & 24.2 & 10 & 0 & 9 & 13\\ \hline    \hline 
   
     \hline
  \end{tabular}
  \caption{Characteristics of applications\label{table:characteristics}}
\end{table}
\subsection{Memory Overhead}
\label{sec:memory}

We will evaluate the performance overhead of the profiler itself. 

\begin{table}[!tp]  
\centering    
\caption{Memory consumption of Linux's Default with MMProf \label{tab:memory_consumption}}    \begin{tabular}{|l|r|r|}    
\hline    
Applications &  Default  & With \MP{}\\ \hline   
   blackscholes &628681 &1011985\\ \hline    bodytrack& 33070&213612 \\ \hline    cache-scratch &3450 &152028\\ \hline    cache-thrash&3896&155926\\ \hline    canneal&872241&1934704\\ \hline    dedup&1194100&2393497\\ \hline    facesim&322069&714020\\ \hline    ferret&125513&346869\\ \hline    fluidanimate&231920&558696\\ \hline    freqmine&3513129&5904754\\ \hline    histogram&1376432&1509108\\ \hline    kmeans&22538&203553\\ \hline    larson&345286&435672\\ \hline    linear_regression&5830226&5963057\\ \hline    matrix_multiply&50161&197894\\ \hline    pca&502980&827778\\ \hline    raytrace&1317749&2198701\\ \hline    reverse_index&1147026&1842073\\ \hline    streamcluster&114602&315606\\ \hline    string_match&1636385&1769350\\ \hline    swaptions&7676&160146\\ \hline    threadtest&524588&1142986\\ \hline    vips&94312&283276\\ \hline    word_count&3129&729449\\ \hline    x264&1029130&1178769\\ \hline    \hline 
      Total&{\bf 20930289}&{\bf 32143509}\cr\hline    
   \end{tabular}\end{table}



\begin{comment}

\subsection{Range of Allocator Metrics}
We will provide the metrics to evaluate the allocators, based on the averaged value. 
\todo{What types of metrics should we used? For instance, what type of policy should we used to exclude an allocator, and then get the value of the allocator. 20\%}
We will provide a table that can be utilized to evaluate all future allocators. 


%Jin


\end{comment}

