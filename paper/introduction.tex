

\section{Introduction}

The memory management has a significant performance impact on the performance of applications, which could lead to as large as $9.7\times$ performance difference with different allocators. Figure~\ref{fig:motivation} evaluates the performance of five applications from two popular benchmark suite, PARSEC-2.0~\cite{parsec} and Phoenix~\cite{phoenix}, with the default Linux allocator (Glibc-XXX), TCMalloc-2.2, jemalloc-4.2.0, and Hoard-3.11. The figure shows that choosing a different allocator may cause the performance difference between 32\% faster to 9.7$\times$ slower.   

\begin{figure}[!ht]
\centering
\includegraphics[width=3.3in]{figures/motivation}
\caption{Performance with Different Allocators\label{fig:motivation}}
\end{figure}

Facing with this vast performance difference of different allocators, programmers may want to choose an appropriate allocator for their applications, in order to maximize the performance potential. Also, when facing performance degradation, they may also want to identify whether the degradation is caused by the allocator or not. For allocators designers, it is also extremely helpful to know the underlying reason of the performance slowdown (or speedup) of a specific allocator, so that they could further augment their design. All of these targets could be satisfied with an allocator profiler.  

Unfortunately, there is no such profiler to the best of our knowledge. Existing allocation profilers, such as \texttt{mprof}~\cite{Zorn:1988:MAP:894814}, \texttt{TCMalloc} Profiler~\cite{tcmalloc-profiler}, CLR profiler~\cite{lupasc2014dynamic} or others~\cite{hirotaka2003developing}, typically focus on memory allocation behavior of applications. For instance, \texttt{mprof} breaks down  memory allocations based on the callstack~\cite{Zorn:1988:MAP:894814}. The \texttt{TCMalloc} Profiler reports program sites with a large number of allocations, locates memory leaks, and reports heap usage of any time~\cite{tcmalloc-profiler}. In addition, general profilers, e.g. gprof~\cite{DBLP:conf/sigplan/GrahamKM82}, Coz~\cite{Coz}, and perf~\cite{perf}, are not suitable for profiling memory allocators as well, since they typically do not differentiate the behavior caused by allocators or applications.  
 
The lack of a general profiler typically makes  allocator developers design their custom profilers every time, wasting their effort each time. Also, it is impossible to quantitatively compare different allocators. Existing comparisons~\cite{Barroso:1998:MSC:279358.279363, Masmano:2006:CMA:1167999.1168012, ferreira2011experimental}, mostly just evaluate the performance of a set of applications, without presenting the underlying reason behind the performance. 
%Therefore, they could not guide the choice of allocators well, since the best performance on these applications may not be good for a specific applications. 
%Even if they find out that one allocator is not good at the performance by running a serial of applications, they cannot further pinpoint what issues inside the allocators. Therefore, they cannot guide the allocator designer to further fix the issues inside, or prevent such issues in the future design. 


This paper, \MP{}, designs the \textbf{first} general profiler to evaluate the allocation/deallocation behavior of allocators. \MP{} focuses multiple aspects of the allocator, such as performance overhead, memory overhead, scalability, and applications friendliness. The first three aspects focus on the allocator itself. The application friendliness evaluates the impact on the performance of applications, e.g. whether an allocator could reduce cache misses, TLB misses, or unnecessary remote NUMA accesses. These aspects are combined together to evaluate an allocator quantitatively. 

For performance overhead, \MP{} not only evaluates the time spent in each allocation and deallocation, but also collects the hardware events of each allocation and deallocation, e.g. the number of instructions, cache misses, and TLB misses. Presenting the information based on each allocation and deallocation, instead of report a sum of these events, helps identify some particular issue inside the OS intuitively. For instance, \MP{} identifies that DieHarder causes excessive number of cache and TLB misses upon each deallocation, e.g. \todo{5 times} on average,  which clearly indicates an significant performance issue inside~\cite{DieHarder, Guarder}. \MP{} divides the memory overhead of an allocator into separate parts, such as metadata overhead, alignment overhead (internal fragmentation), and memory blowup, which allows programmers to further analyze the issues of their allocators. For instance, memory blowup is defined as the unnecessary increase of memory consumption when an multithreaded allocator reclaims the memory but fails to use these freed objects to satisfy future memory requests, typically from another thread~\cite{Hoard}. Most existing allocators utilizing the ``per-thread'' heap to reduce the lock contention, but will increase the memory blowup significantly. For the scalability, \MP{} focuses on the quantification of both user space contention and kernel space contention. Note that an allocator may invoke different system calls, such as \texttt{mmap}, \texttt{brk}, \texttt{madvise}, and \textbf{munmap}, which may cause unnecessary kernel contention inside the OS (and limit its scalability). As described before, \MP{} further evaluates the application friendliness of each allocator. For application friendliness, we propose multiple metrics like cache utilization ratio, page utilization ratio, and the number of ownership changes to evaluate the cache friendliness, TLB friendliness and NUMA friendliness.     


The implementation of \MP{} faces multiple challenges.

 
Challenge 1: How to know the specific details of different allocators? We utilized a small program to get the allocator's specific feature. For instance, whether they are BIBOP style or Bump-pointer based, the size class information and the metadata information. 

Challenge 2: how to perform the profiling? Similar to existing work, we majorly use the time (supported by RTDSC), the number (instrumentation-based counting), and some hardware events (PMU events) to perform the sampling. The sampling approach will be similar to existing work, but we attribute those events to the memory management events, such as allocations and deallocations. 

Challenge 3: how to reduce the performance overhead? In order to reduce the number of cache contention, we re-design our data structure to avoid false sharing and true sharing as much as possible. Also, we 

Challenge 4: we propose a novel method to evaluate the application friendliness. We evaluate the cache friendliness, or TLB friendliness. 

Challenge 5: we employs an internal allocator to avoid the interfering with allocations and deallocations of applications.  


Overview:
\MP{} is an drop-in library that should be linked before any runtime library. Similar to existing profilers, \MP{} also collects  hardware events, time information and the number of invocations. However, \MP{} attribute these events or data to each invocation of memory allocation and deallocation, which can present users  intuitive information about the possible issue of each allocator. As a profiler, \MP{} further summarizes the performance overhead, memory overhead, scalability, and application friendliness of each allocator. 


For performance overhead, \MP{} focuses on the following items: 


Since memory overhead can be caused by multiple factors, such as metadata overhead, alignment overhead (internal fragmentation), and memory blowup. Memory blowup is defined as ,. 

For scalability, \MP{} not only focuses on the potential bottleneck within the user space, but also includes the potential bottleneck caused by the allocator. Programmers have identified one particular performance bottleneck of applications, such as dedup issue. 
It won't cause the . 

\MP{} also aims to answer whether an allocator is friendly to the applications or not, such as . . 


   


MallocProfiler will serve two purposes. 
(1) This will be utilized to get the allocator related parameters. 

(2) We will be able to identify whether the performance problem is caused by the memory allocator. We could actually divide the overhead to the overhead of allocator or the overhead of applications. 
Therefore, we could identify the root causes and not always blame for the application writer. 



 

However, the memory allocator itself has not got sufficient attention that it should reserve to. 
For instance, there is no an dedicate the allocator profiler that can be utilized for identify the allocator's behavior.

Having this profiler will serve three purposes. First, it will help the designers and developers that could discover the issues of memory allocators, without the need of porting or developing the profiler for a specific allocator. Second, it helps programmers to determine whether the performance issue is coming from the memory allocator. Third, it will help users to choose the best memory allocators that is suitable for a specific applications, if there are multiple choices available. 

 Linux, TCMalloc, Jemalloc, Hoard, OpenBSD, DieHarder. Then shows the internal reason of why some are slower than others. 

We may choose two general-purpose allocators and two secure allocators. Then we know why secure allocators are slower, currently. 

The general idea is to understand performance, memory, and scalability of different allocators, while providing some evidence of allocator. During the design of allocator, we usually need to understand the inherent reason. The difficulty is how to collect the information without changing the specific allocator. That is, how to design a general allocator profiler, which can help assist the analysis of different allocators. 

We will work on the allocator profiling to understanding the performance, memory and scalability of memory allocator. We may  memory related those system calls. How long it spent on the memory allocation and deallocation, using rtdsc, which can be caused by long memory allocation time? How much of objects has been re-utilized? How is the memory blowup, what is the memory consumption and how much has been allocated? Can we know lock contention of memory allocation, maybe we could monitor the lock usage, just as SyncPerf, and identify that lock acquisitions are  waiting during allocation? Can we identify is there inter-objects cache contention on these objects? If yes, that is the possible example of these allocators.  

What are the design goals of \MP{}? 

\MP{} needs to be general to different memory allocators, which requires no change of the code at all even when connecting with different allocators. The first target is transparency, which should not require the changes of allocators, if the allocator is working as a dynamic library. 

Second, \MP{} should be able to identify the issues of different allocators. 




\begin{comment}

Dynamic memory management plays an important role in the performance of applications, especially on multithreaded programs. 
For performance related to memory uses, some work focuses on improving existing memory allocators. Some focuses on a better memory layout among different elements of the same data structure. 
But there is little work that focuses on the improving the performance by changing the behavior of memory allocations and deallocations. 

\HeapPerf{} tries to identify some places inside applications that can introduce the performance problems. These problems can be solved by changing the behavior of memory allocations, without using the new memory allocator. 
These problems are rarely investigated in the past. The most closest work related to this is to simply record the placement with excessive allocations. However, excessive allocations is a total class of the problems that are investigated in this paper, but the existing work fails to present any detailed idea on the following problems: whether all these excessive memory allocations can be reduced? whether they can improve the performance? how to reduce that? These questions requires highly expertise and large amount of manual effort. Instead, \HeapPerf{} presents more useful idea on these questions, and hope to guide programmers, even non-experts, to solve these problems easily. 

We observe three different patterns that can cause performance problems, or called as anti-patterns~\cite{}. 

The second type is shown as Figure~\ref{}. In this example, there are a number of memory allocations that will allocate small amount of bytes for each one. More particularly, these allocations are inside the same loop, and have the same size. Unfortunately, memory allocators will not precisely allocate the specified size of objects. For example, the \texttt{glibc} allocator will allocate 32 bytes as long as the required size is between 4 bytes and 24 bytes. Based on the explanation of Hoard~\cite{Hoard}, this method helps to manage small sizes of different objects, without introducing too many external fragmentation. However, this also introduces a significant problem on cache inefficiency, since only less than 13\% cache is actually utilized, which can introduce around $8\times$ performance slowdown comparing to the code listed in Figure~\ref{}. 


The second type is shown as Figure~\ref{}, there are a number of unnecessary memory allocations and deallocations. By moving the placement of allocations to outside the loop, we can significantly improve the performance by reducing the overhead related to memory allocations and deallocations. 

The third type is related to the uses of heap variables or stack variables. Some excessive heap objects, if they are turned into stack variables, will have large performance benefit. Comparing to heap objects, the overhead of memory allocations and deallocations can be largely reduced if using stack variables. Also, the stack is typically locate inside the cache, which will have lower access latency. Also, stack variables will have exact size, without the addition of metadata and huge alignment. 
\todo{Whether those variables have been touched only very few times, typically should be putted into the heap objects since they may cause the in-efficient cache utilization as well}.  
 


Heap memory related performance bugs can be from the following categories, if only think about applications. 

\begin{itemize}
\item: Too many allocations and deallocations. The total run time spending in memory allocation and liberation may take up to 30\% execution time\cite{1190248}. 

\item: Too many memory uses: this can actually affect the performance when there are too much memory that has been allocated but not used. This can be caused by memory leaks or too late de-allocation. \todo{Most existing tools focus on the memory leak, but whether there are some tools that can uncover delayed-deallocation?}  

\item: 
\end{itemize}


What we can do for heap memory management?

First, we can point out the unnecessary memory allocations and deallocations. For example, we can malloc a large object, and then assign to different small projects. Some of them may be called inside the internal level of loop functions, we can move up to external loop level. 
By reducing unnecessary memory allocations, we expect to improve the performance. 

Second, we can give a statistics on the life-span of objects. Whether we can find out some problems inside? For example, we can use stack variables instead of heap if some objects are too short-lived, or mostly inside a function call. 

Third, we can actually give the statistics on each callsite. Some callsites may have larger number of allocations. 

Can we evaluate the performance related to heap allocations? For example, how much time is spending on memory allocation. 
We can approximate the time of spending on each allocation. Then we can attribute the time to different statements, just similar to gprof. Then maybe it is obvious that we can reduce the overhead by reducing the memory allocations. Then it is possible that a separate paper by using the 


In the end, although not every interested, it is to check the overhead of every memory allocation on each popular memory allocation. Thus, pointing out that the memory allocation actually should pay attention to the level of stacks. Thus, it is possible that we can design a new memory allocator by reducing the level of memory allocation. This is a reverse to HeapLayer. It is great to have an survey paper on this:

A. How is the overhead of memory allocation in large applications? How we can evaluate it? 
B. How is the overhead that comes from memory management? We evaluate this on some popular benchmarks. 
C. Whether the overhead comes from different cache uses? or other things. 
D. It will shed a light whether we need to re-design the memory allocator. 
It will be a perfect paper for ICSE or SC.

How we can evaluate the cache friendliness of memory allocators? 
For instance, how much memory will be reutilized immediately? If one direct use will be one point, then how many score of different allocators. 

Also, how much of memory blowup? 

%%%%%%%%%%%%%%%%%%%%%%%
% Possible solutions:
% (1) We will check the malloc and free are allocated in sequence. For example, we are always doing the malloc(8) and free(8) in sequence. Given the number of these allocations is large. Then it is much possible that it is a problem. However, it can be a problem for the performance reason. But we can basically maintain a stack that maintains five possible allocations. 
%% Should we just use a two-phase solution? That is, we can use a hash-table to identify different allocation site with their memory uses: how many times for memory allocations? How many times for related free operations? If there are a lot of memory allocations that are not freed, then it is possible a memory leak. We could also identify whether those memory are actually touched or not by using the watchpoint mechanisms? Also, we may try to check whether memory allocation are in the same sequence, for example, alloc-free-alloc-free, and with the same size. If yes, then it is possible that is unnecessary memory operations. 

%%%%%%%%%%%%%%%%%%%%%%%%%
Typically, I think that mtrace utilizes 

Can we check the example of malloc-free situations?
Can we base on a ``anomaly detection'' but.  
I guess that the memory will be freed before the next allocation. If not, then there is high probability of leaking. If memory allocated is on the same site, 
	
\end{comment}
 
\subsection*{Contribution}

This paper makes the following contributions. 

\begin{itemize}
\item It designs and implements \textbf{the first general profiler}--\MP{}--to profile different aspects of different memory allocators, without changing allocators themselves.  

\item \MP{} employs advanced hardware, such as Performance Monitoring Units (PMU) or Read Time-Stamp Counter (RDTSC), and the simple counting together to profile allocators quantitatively. 
%Overall, it can not only pinpoint the major issues inside the implementation of memory allocators, such as performance, memory, and scalability issues, but also could quantitively pinpoint cache or page utilization issues of allocators.  

\item \MP{} proposes multiple metrics to evaluate application friendliness. Further, it proposes the employment of PMU support to sample these parameters efficiently and effectively. 

\item This paper performs extensive experiments on multiple memory allocators. It pinpoints some performance or memory issues latent in existing allocators. It also provides the first quantitative comparison of different factors of these allocators.  

\end{itemize} 

\subsection*{Outline}

The remained of this paper is organized as follows. Section~\ref{} discusses the overall design purpose of \MP{}, and Section~\ref{} presents the detailed implementation. Section~\ref{} shows the results of experiments on different allocators using \MP{}. Then Section~\ref{} discusses related work in this field, and Section~\ref{} concludes this paper. 