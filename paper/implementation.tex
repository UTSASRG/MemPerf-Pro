\section{Detailed Implementation}

\section{Implementation}


\input{implement-stefen}
\input{implement-sam}

\subsection{Understanding Different Allocators}

Generally there are two types of allocators, e.g. pump-pointer based and BIBOP-style allocator. For BIBOP-style (``Big Bag of Pages'') allocators~\cite{hanson1980}, one or multiple continuous pages are treated as a ``bag'' that holds objects with the same size class.  The bump-pointer allocators typically utilize a pointer pointing to the starting of never-allocated objects, which will update the pointer to next position after each allocation.   


How to identify the type of allocators? Typically, most allocators only utilize BUMP pointer based allocators or BIBOP-style allocators. 

\subsubsection{Types of Allocators}
BIBOP style is the abbreviation of ``Big Bag of Pages''~\cite{hanson1980}.
For BIBOP-style allocators, one or multiple continuous pages are treated as a ``bag'' that holds objects of the same size class. The metadata of each heap object, such as its size and availability information, is typically stored in a separate area.

Bump pointer
How to collect the size class information?
How to solve the issue when connecting to some allocators, such as Guarder, since Guarder interposes the thread execution. 

How to intercept the memory allocations and deallocations? 

Region-based allocators: 
it is not selected in this study, since we can't find many existing allocators with a high performance. 


\subsection{Profiling Performance Overhead}

We differentiate the big objects and small objects. 

\subsection{Profiling Memory Overhead}

\subsection{Profiling Scalability}

\subsection{Application Friendliness}

\subsection{Optimizations}

\subsubsection{Designing the Fast Lookup}

\label{sec:fastlookup}

\MP{} is designed to adapt to different allocators, which may have different mechanisms to obtain the memory from the OS, such as via \texttt{brk} or \texttt{mmap} system calls separately. Therefore, memory address spaces are scattered along the whole address space of a process, which makes difficult to utilize virtual memory to provide a fast lookup. Based on our observation, \texttt{mmap} system calls always allocate the space starting from 130 TB space (even with ASLR mechanism enabled). Therefore, \MP{} reserved the last 1TB space for the shadow memory of this.   
