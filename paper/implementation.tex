\section{Detailed Implementation}

\section{Implementation}The profiler relies on information that it's ``prerun'' library provides it. The prerun library determines the allocator's style (bibop, or bump-pointer), class sizes if bibop, a size threshold where malloc will use mmap to satisfy the allocation, and an approximate size of metadata per object. Getting the class sizes for bibop allocators is tricky. The prerun allocates an object stating at 8 bytes, and continues to realloc the objects address with the objects size plus 8 bytes. We discover the class size once the object is moved to a different location, or class size bag. Open BSD's allocator ``reverse aligns'' objects that are between 2Kb and 4Kb in size. The allocator aligns the objects last byte to the end of the page in which it resides, so that the object starting byte is somewhere in the middle. The prerun would see the address change and believe it has found a new class size. Open BSD provides a flag to turn this behavior off, however this is one example of the challenges of making the profiler general purpose.

	There are many challenges in searching for metadata. Different allocators employ completely different methods in storing metadata for it's internal use. There are a number of ways we attempt to get metadata size for each object. For a bibop allocator, it's metadata has a good chance of being located in a private memory region returned from mmap. The prerun detects the mmap and checks to see if it was used to fulfill a memory request. If it wasn't we will search it for metadata. The prerun library utilizes the /proc/pid/pagemap file to identify the virtual pages which are backed by physical frames. The prerun searches the entire mmap region for physically backed virtual pages, remembering their page numbers in relation to the mmap region starting address. Finally, the prerun counts the number of non-zero bytes on all physically-backed virtual pages, using the minimum of all these values as an estimated metadata size. This method works with the assumption that the bibop-style allocator \emph{is} placing it's metadata in mmap memory. One of the allocators we tested, Dieharder, does mmap memory that was not used for allocations nor metadata, however the prerun includes these regions in it's search for metadata incorrectly. Additionally, the actual metadata is interpreted differently for different allocators. One allocator can store primitive data types in it's metadata, while another allocator uses bitmaps or byte values that have specific meaning only to that allocator. The numerous ways of storing and interpreting metadata is a huge challenge for us.
		
			Another huge challenge in the profiler is globally storing data that is thread specific in a way that is quickly accessed and synchronized between threads. There's no getting around all threads needing to access some type of global data. Doing this efficiently is a priority for the profiler. Detecting memory blowup requires the allocating thread to be aware of other threads free object status. An internal global free list of objects has too much contention from threads simultaneously inserting, deleting, and searching the free list. We employ a semi-thread local approach. Each thread has a thread local count of free objects (by class size for bibop) and the profiler maintains one global count (again by class size for bibop). When a thread allocates, there are three possibilities that we check in determining memory blowup, always starting with the thread checking it's thread local free count. If the thread has free objects it can reuse, the profiler assumes that the thread will reuse one of it's free objects, and decrements its local and global counter. If the thread does not have free objects, it checks to see if there are free objects globally. If there are, the profiler accounts for that one occurrence of memory blowup and decrements the global counter. The last situation is that the thread does not have any free objects, and the global counter is at zero. There is no memory blowup in this situation. The global counter is synchronized using atomic accesses. This is accurate and fast method the profiler uses in detecting memory blowup. Balancing the profiler with accurate synchronized global data and performance is a huge challenge.


\input{implement-stefen}
\input{implement-sam}

\subsection{Understanding Different Allocators}

Generally there are two types of allocators, e.g. pump-pointer based and BIBOP-style allocator. For BIBOP-style (``Big Bag of Pages'') allocators~\cite{hanson1980}, one or multiple continuous pages are treated as a ``bag'' that holds objects with the same size class.  The bump-pointer allocators typically utilize a pointer pointing to the starting of never-allocated objects, which will update the pointer to next position after each allocation.   


How to identify the type of allocators? Typically, most allocators only utilize BUMP pointer based allocators or BIBOP-style allocators. 

\subsubsection{Types of Allocators}
BIBOP style is the abbreviation of ``Big Bag of Pages''~\cite{hanson1980}.
For BIBOP-style allocators, one or multiple continuous pages are treated as a ``bag'' that holds objects of the same size class. The metadata of each heap object, such as its size and availability information, is typically stored in a separate area.

Bump pointer
How to collect the size class information?
How to solve the issue when connecting to some allocators, such as Guarder, since Guarder interposes the thread execution. 

How to intercept the memory allocations and deallocations? 

Region-based allocators: 
it is not selected in this study, since we can't find many existing allocators with a high performance. 


\subsection{Profiling Performance Overhead}

We differentiate the big objects and small objects. 

\subsection{Profiling Memory Overhead}

\subsection{Profiling Scalability}

\subsection{Application Friendliness}

\subsection{Optimizations}

\subsubsection{Designing the Fast Lookup}

\label{sec:fastlookup}

\MP{} is designed to adapt to different allocators, which may have different mechanisms to obtain the memory from the OS, such as via \texttt{brk} or \texttt{mmap} system calls separately. Therefore, memory address spaces are scattered along the whole address space of a process, which makes difficult to utilize virtual memory to provide a fast lookup. Based on our observation, \texttt{mmap} system calls always allocate the space starting from 130 TB space (even with ASLR mechanism enabled). Therefore, \MP{} reserved the last 1TB space for the shadow memory of this.   
