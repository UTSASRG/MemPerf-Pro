\section{Detailed Implementation}

\label{sec:implement}

\section{Implementation}


\input{implement-stefen}
\input{implement-sam}

\subsection{Understanding Different Allocators}

\label{sec:understandingallocators}


\subsection{Profiling Performance Overhead}

We differentiate the big objects and small objects. 

\subsection{Profiling Memory Overhead}

\subsection{Profiling Scalability}

\subsection{Application Friendliness}

\subsection{Optimizations}

\subsubsection{Designing the Fast Lookup}

\label{sec:fastlookup}

\MP{} is designed to adapt to different allocators, which may have different mechanisms to obtain the memory from the OS, such as via \texttt{brk} or \texttt{mmap} system calls separately. Therefore, memory address spaces are scattered along the whole address space of a process, which makes difficult to utilize virtual memory to provide a fast lookup. Based on our observation, \texttt{mmap} system calls always allocate the space starting from 130 TB space (even with ASLR mechanism enabled). Therefore, \MP{} reserved the last 1TB space for the shadow memory of this.   
