%\section{Motivation Example}
%\label{sec:motivation}

%Existing profilers cannot help identify some performance issues caused by the memory allocator. Let us use \texttt{cache-thrash} shown in Figure~\ref{fig:motivation} as an example. For this application, TcMalloc runs around $48\times$ slower than the default Linux allocator. We are using perf, gprof, and Coz to analyze the performance issue. 
 
%The perf's result is shown in Figure~\ref{fig:mot1}. It reports that over 99.81\% of time is actually spent inside the \texttt{worker} function, without pinpointing the real issue. Similarly, the gprof tool actually reports that 100\% of time is spent inside the \texttt{worker} function. Coz reports the program lines of exercising the objects with passive false sharing issues, i.e. lines 85-87 of \texttt{cache-thrash.cpp}, and predicts that the performance can be improved up to \textbf{10\%} if these lines can be improved by 100\%. But there are two issues. First, the impact is obviously much smaller than the real one. This will mislead the programmers. Second, programmers may not know how to improve the performance for three reported statements, since Coz does not pinpoint the real reason for the slowdown, which are passive false sharing issues caused by the corresponding allocators. In summary, existing profilers are not very useful to figure out most performance issues caused by an allocator.   


%For this application, TcMalloc actually introduces both active and passive false sharing issue, which is the major cause for this large slowdown.  Similarly, Coz also cannot identify the issue that are caused by hardware contention~\cite{DBLP:conf/osdi/ZhouGMW18}. For this example, the first line of the code that can be improved by $XX$ is actually located in line $XX$ of \texttt{} file. That is also nothing related to the allocator.  
%In contrast, \MP{} reports this problem as passive false sharing issues caused by the allocator, with one parameter inside its application-friendliness metrics. \MP{} also reports a range of metrics for evaluating an allocator, such as other cache friendliness metrics or memory overhead, where those metrics are useful for diagnosing the performance and memory issues caused by an allocator.     

\section{Background and Overview}
\label{sec:background}

This section presents some background of memory allocators, and then describes the basic idea of \MP{}. 

\subsection{Background of Allocators}

\label{sec:allocator}
Memory allocators are typically responsible for managing virtual memory inside the user space by satisfying memory requests from applications. Since the number of small objects is significantly larger than that of big objects,  allocators typically utilize different mechanisms to manage small and big objects~\cite{Hoard}. Small objects are typically managed by size classes, so that every allocation request will be rounded to its next largest size class, which helps reduce external fragmentation and encourage memory utilization. In order to reduce potential contention, modern allocators are typically utilizing per-thread freelists to track objects freed by the current thread, which will introduce memory blowup issue~\cite{Hoard}.

Based on this knowledge, \MP{} designs a pre-run program to collect the threshold of big objects and small objects, and collect class size information for small objects. All these information will be stored into a file that will be utilized by \MP{}, where users could also provide such a file for an allocator. Based on this information, \MP{} will collect the data for big and small objects separately, in order to identify the specific design issues inside. The information of size class is able to help identify internal fragmentation. Due to per-thread design, it is very important to identify memory wastes caused by memory blowup. 

 
%For big objects, allocators may obtain a block of memory from the OS directly during the allocation, and then return it to the OS upon the deallocation~\cite{Hoard}. For small objects, allocators may utilize freelists or bitmaps to track freed objects upon deallocations. 

%Based on the management of small objects, allocators can be further classified into multiple categories, such as sequential, BiBOP, and region-based allocators~\cite{DieHarder, Gay:1998:MME:277650.277748}. Region-based allocators are those allocators that all allocated objects within the same region are deallocated together~\cite{Gay:1998:MME:277650.277748}, which do not belong to general-purpose allocators. For sequential allocators, subsequent memory allocations are satisfied in a continuous memory block. Such allocators include the default Linux allocator (originating from dlmalloc~\cite{dlmalloc}) and the Windows allocator~\cite{DieHarder}. BiBOP-style allocators, which stands for ''Big Bag of Pages''~\cite{hanson1980}, utilize one or multiple continuous pages that are treated as a ``bag'' used to hold objects of the same size class. Many performance-oriented allocators, such as TcMalloc~\cite{tcmalloc}, \texttt{jemalloc}~\cite{jemalloc}, Hoard~\cite{Hoard}, Scalloc~\cite{Scalloc}, and most of secure allocators, such as OpenBSD~\cite{openbsd} and DieHarder~\cite{DieHarder}, belong to this category.

%Typically, a pointer is utilized to track the starting position of available space~\cite{Cling}. After an allocation, the pointer is bumped to the end of the current object, which is why this class is also known as ``bump-pointer allocators.'' For such allocators, objects with different sizes can be allocated continuously. Upon deallocation, a freed object is typically placed into the freelist of its size class. The size information of each object is typically physically placed just prior to the object.   

%The metadata of these heap objects, such as their size and availability information, are typically stored in a separate area. Thus, BiBOP-style allocators improve security and reliability by avoiding metadata corruption caused by buffer overflows.  BiBOP-style allocators may utilize freelists or bitmaps to manage the availability of objects. 
%When using a bitmap, only a single bit is sufficient to track the availability of an object, which may introduce less memory overhead for the metadata, but possibly with a higher performance overhead due to the manipulation of the bitmap and the loss of temporal locality.  

\begin{comment}

\subsection{Important Metrics of Allocators}

\label{sec:factors}

Every memory allocator has its own design choices. However, they share many similar factors, such as performance, memory, scalability, and application-friendliness characteristics. This section will list important metrics for these factors that will also be reported by \MP{}. 

\subsubsection{Performance}
\label{sec:performance}

The performance of using a memory allocator depends on two aspects, the performance overhead of its memory management operations, as well as its application-friendliness toward a specific application (as discussed in Section~\ref{sec: friendliness}). We will focus on the former item here. 

The performance overhead can be evaluated using the average number of instructions and the average runtime for each allocation and deallocation operation. The average data is more intuitive and understandable than the summary value alone. As described above, a memory allocator has different execution paths for different types of allocations and deallocations. Therefore, \MP{} further differentiates the type of memory operation during profiling, such as new or re-used allocations for small objects, deallocations for small objects, and allocations/deallocations for large objects. By doing this, \MP{} is able to identify an issue inside a particular execution path.
%\MP{} relies on a configuration file to obtain the threshold between small and big objects, and . 

In order to reveal a specific design issue, \MP{} further collects the averaged number of instructions, cache misses, and TLB misses for each operation, which are important factors for diagnosing performance issues. For instance, we use the number of cache misses inside DieHarder to diagnose its design issue for its memory deallocations. 
%a large number of instructions possibly indicates an inefficient design of an allocator. 

\subsubsection{Memory Consumption}
\label{sec:memoryconsumption}

Memory consumption is a serious concern across different platforms. Therefore, it is important to assess the memory consumption and wastage of a memory allocator. Sometimes, an allocator may waste more memory than that caused by memory leaks within an application. Based on our understanding, the memory consumption of an allocator may be attributed to multiple sources. First, it may originate from the metadata, such as the memory used to track the size or availability of every heap object. 

Second, it may be caused by internal fragmentation due to the use of size classes.  The difference between the requested size and the size of the corresponding class represents its internal fragmentation, and this space cannot be utilized to satisfy other allocation requests. For instance, Hoard manages objects using power-of-two size classes~\cite{Hoard}, which thus may waste memory if the requested size is not an exact power-of-two. 
%As discussed above, memory allocators utilize multiple size classes to manage heap objects rather than using an exact size.
 
Third, memory allocator wastage may come from ``memory blowup.'' Memory blowup occurs when memory deallocations from one heap cannot be utilized to satisfy subsequent memory requests from another thread~\cite{Hoard}. 
%In order to solve this issue, Li et. al. employ heuristics to adjust the synchronization frequency dynamically~\cite{DBLP:conf/iwmm/LiLD19}. 
Since modern allocators typically utilize per-thread heaps or multiple arenas to reduce the contention overhead, memory blowup is a major source of memory consumption. However, it is challenging to actually quantify memory blowup, as further described in Section~\ref{sec:profilingmemory}.   

Fourth, memory consumption may come from external fragmentation. External fragmentation occurs when the total amount of the available memory is sufficient to satisfy a request but fails to do so due to  non-contiguous memory. External fragmentation is also related to the use of size classes, 
%because allocators rarely or never perform object coalescing and splitting.
since it is typically impossible to change the size class.  
%for a few objects inside the bag, since BiBOP-style allocators typically use a single size for all objects in the entire bag. 
%That is, objects cannot be changed to other size classes, until all objects in the whole bag are freed. 
%This design may cause extensive external fragmentation overhead. 

 %Lastly, some secure allocators may voluntarily skip certain objects in order to tolerate buffer overflows~\cite{DieHard, DieHarder, Guarder}. If a buffer overflow lands within these non-used objects, it will cause no harm to the application. However, it is extremely difficult to differentiate between explicit skipping and external fragmentation, with the effects of both ultimately being identical with regard to their impact on this type of memory overhead. 

Overall, for memory consumption, \MP{} individually reports the number and ratio of real memory usage, internal fragmentation, memory blowup, and other overhead. Other overhead includes external fragmentation, metadata overhead, and the sum of skipped objects. It is difficult for \MP{} to possess the precise information about the metadata and the sum of skipped objects. Therefore, \MP{} reports a summary for other memory overhead by subtracting the mentioned values from the total memory usage. For the total memory usage, \MP{} tracks memory-related system calls (e.g., such as \texttt{mmap} or \texttt{sbrk}). 
%Overall, \MP{} provides real memory usage of t
%\MP{} also reports real memory usage by tracking memory allocations and deallocations.  


\subsubsection{Scalability} 
\label{sec:scalability}

The scalability of an allocator can be affected by both hardware and software contention. Hardware contention is mostly related to cache or page contention, which is discussed in Section~\ref{sec: friendliness}. Software contention is the focus here, further including user space contention and kernel contention. 

\paragraph{User Space Contention} The user space contention of an allocator is typically caused by the use of locks inside memory management operations. Based on our observation, different allocators have significantly differing behaviors regarding lock usage. Some allocators, such as TcMalloc~\cite{tcmalloc} or jemalloc~\cite{jemalloc}, minimize the use of locks via per-thread cache. If an allocation can be satisfied from a per-thread cache, there is no need to acquire a lock. However, some allocators, such as Hoard~\cite{Hoard}, acquire at a lock for each allocation request, although using its per-thread heap.p. Some allocators, such as DieHarder and OpenBSD, utilize a central heap (and lock) for each size class, causing too much contention. 

%The average time for each lock acquisition indicates potential lock contention inside. The average time of each critical section helps expose whether the lock contention is due to the heavy workload inside the critical section or not. For instance, if the contention is high, but the average time inside the critical section is low, then this allocator should employ more fine-grained locks to distribute its overhead. In contrast, if the average time inside the critical section is high, then the allocator should possibly move some computation out of the critical section or simplify its management. 

To evaluate user space contention, \MP{} collects per acquisition data, per-operation data, per-lock data, and total information of locks. Per-acquisition data includes the runtime of each lock acquisition and the runtime of each critical section. For each operation, \MP{} reports the number of locks and the average runtime of each acquisition for different operations, such as new small allocations, re-used small allocations, small deallocations, and large allocations and deallocations. Further, \MP{} reports the total number of locks used inside the allocator, then reports the number of acquisitions, the contention rate, and the number of acquisitions for each operation for every suspicious lock that may have a contention issue. 
 
\paragraph{Kernel Space Contention} 
 An allocator may introduce kernel contention by invoking memory-related system calls frequently, such as \texttt{mmap}, \texttt{munmap}, \texttt{madvise}, and \texttt{mprotect}. These system calls may conflict with each other and with the page fault handler. By examining the source code of the Linux kernel, they all acquire a process-based lock (e.g. \texttt{mmap\_sem}) upon the entry of these system calls, causing kernel contention. Based on our evaluation, a version of the Linux allocator slows down an application by more than 20\%, due to extensive invocations of the \texttt{madvise} system call. Therefore, it is important to measure kernel space contention caused by a memory allocator. \MP{} proposes to utilize the average runtime of each system call to evaluate potential kernel contention, without the need to change the kernel source code.

For kernel contention, \MP{} reports the average time and the number of invocations for each memory-related system call. Since different operations (e.g., small versus large allocation) have different execution paths, \MP{} further reports the number for each particular operation. This differentiation helps identify an issue that may only appear in a particular operation pathway. 

\subsubsection{Application Friendliness}
\label{sec: friendliness}

Application friendliness indicates whether memory allocations are suited toward the access patterns of a particular application. Sometimes, application friendliness may have a larger impact on an application's performance than its memory management performance. For instance, TcMalloc typically has less memory management overhead than the default allocator, but runs around $48\times$ slower for \texttt{cache-thrash}. The major reason for this is that TcMalloc introduces both active and passive false-sharing~\cite{tcmallocsharing}. \MP{} reports multiple important metrics that evaluate application-friendliness.


The first parameter measured is the cache utilization rate. The cache utilization rate is the percentage of words that are currently holding actual objects. An allocator with a high cache utilization rate will cause less cache misses, benefiting the overall performance. Multiple causes may affect the cache utilization rate. First, some allocators (e.g., the Linux allocator) that prepend the metadata just prior to each object may reduce the cache utilization rate. Every cache load operation will load the metadata that is not referenced during normal memory access. 
 Second, a coarse-grained size class may also harm the cache utilization rate, due to internal fragmentation. Third, freed objects that are not reutilized in a timely manner may also cause a lower cache utilization rate. 

 %Similarly, if page utilization rate is low, it may cause high TLB misses and prohibitive memory consumption. \MP{} samples memory accesses, and checks the corresponding cache utilization rate and page utilization rate. Overall, \MP{} could report an average cache utilization rate and page utilization rate over all samples. 

The second parameter reported is the page utilization rate. The page utilization rate indicates the percentage of pages that are actively utilized for holding actual data. An allocator with a higher page utilization rate will introduce less page faults and less Translation Lookaside Buffer (TLB) misses. Lower page utilization rates can be caused by reasons similar to those that lower the cache utilization rate.  

The third parameter is the active/passive false sharing. False sharing indicates that multiple threads are concurrently accessing different words within the same cache line. Active false sharing is introduced upon the first allocations of objects, where an allocator cannot allocate continuous objects in the cache line to the same thread. Passive false sharing is introduced upon deallocations, where a freed object will be utilized by another thread, causing false sharing within the same cache line. 

The fourth parameter is the cache contention rate outside of the allocation. Cache misses can be caused by conflicting or falsely-shared misses. Upon cache misses, the data has to be loaded from the main memory, which is significantly slower than accessing the cache directly. 

Overall, \MP{} reports the cache utilization rate, page utilization rate, cache contention rate, and false sharing effect. For the false sharing effect, \MP{} not only reports the number of cache lines that have active and passive false sharing, but also reports the rate of conflicting accesses. The reported result also helps explain the performance slowdown issue. 
%employs the PMU hardware to collect these parameters. It employs the PMU-based sampling to sample memory accesses, and collects the data of cache and page utilization data upon sample events, as described in Section~\ref{sec:profilefriendliness}. \MP{} employs PMU to collect cache misses/page faults outside memory management.

\subsubsection{Summary of Important Metrics}
	
\end{comment}



\begin{comment}
\subsection{Allocator's Performance}

\todo{Maybe integrate the previous description}

 What factors of an allocator can affect the performance of the corresponding applications?	

Based on our current understanding, the following factors may affect the performance of applications. 

\paragraph{Allocator's Internal Implementation}

\begin{itemize}

\item The allocator's implementation complexity, or the number of instructions inside the memory management operations. Different allocators may have different complexity. Hoard's implementation is more than twice of TcMalloc for instance for allocating objects with . 

\item The lock contention inside each memory management operation. This can be caused by three reasons. First, an allocator has a wrong threshold of per-thread heap. For instance, Hoard actually has a threshold of evaluating the emptiness of per-thread superblock. Therefore, one deallocation will make Hoard to move the current superblock to the global pool, which is to reduce memory blowup, while the next allocation will be forced to get one superblock from the global pool. Then the global lock will incur lots of lock contention, which slows down an application by $5\times$. Therefore, it is better to have the machine learning to predict the  the working set of each per-thread heap. Therefore, the allocator will not cause too much performance penalty caused by the lock contention, but will still has reasonable memory overhead. Second, an allocator will return some unnecessary memory back to the OS, in order to reduce the memory overhead. However, an allocator (such as the allocator of glibc-2.21) makes the wrong prediction on it, which actually slowdown the application by more than 20\% due to the kernel contention. Third, whether multiple threads are sharing the same heap? If yes, then this is a potential performance issue that can be caused by the lock contention. 

\item Cache performance issue of memory management operation: First, a memory management operation may incur unnecessary cache misses. For instance, DieHard actually causes a significant performance issue caused by cache loading. For each deallocation, DieHard actually incurs a big number of cache misses in finding its corresponding miniheap, since all miniheaps will be checked one after the other. Second, a memory management operation will have some false sharing issues, if multiple threads are accessing the same metadata. Then they can cause some performance issue. 
	Third, it may cause some remote accesses, if we are using the NUMA architecture.
\item Memory allocators incur lot of unnecessary consumption, which may affect the performance due to a large number of page faults, TLB misses, or more cache misses. 
\end{itemize}

\paragraph{Application Friendliness}
\begin{itemize}
\item Cache Utilization Rate: multiple reasons can cause low cache utilization rate, which will incur the in-efficiency of cache utilization. First, if metadata is co-located with the actual object, it may cause low cache utilization rate. Second, if an allocator utilizes too-coarse grained size class, it may cause low cache utilization rate.   
\item Low page utilization rate: similarly, this could affect the performance. 
\item Active/passive false sharing issue: an allocator may introduce active/passive false sharing for applications running on it. 
\end{itemize}
	
\end{comment}



